\begin{thebibliography}{10} \label{bibl}
	\bibitem{abdullin00}
	{\it Абдуллин, Р. З.} Метод сравнения в устойчивости нелинейных разностных уравнений с импульсными воздействиями
	/ Р. З. Абдуллин // Автоматика и телемеханика.— 2000.— № 11.— С. 44-56. 	
	
	\bibitem{ajz747}
	{\it Айзерман, М.В} Основы теории разрывных систем I / М. А. Айзерман, Е. С. Пятницкий
	// Автоматика и телемеханика.— 1974.— № 7-8.— С. 33–47. С. 39–61.
	
	\bibitem{aleksandrov00}
	{\it Александров, В. В.} Оптимизация динамики управляемых систем / В. В. Александров, В. Г. Болтянский,
	С. С. Лемак и др.— М.: МГУ, 2000.— 303 с.
	
	\bibitem{ananev014}
	{\it Ананьевский, И. М.} Управление реономными механическими системами с неизвестными параметрами /
	И. М. Ананьевский // Докл. РАН.— 2001.— Т. 337, № 4.— С. 459–463.
	
	\bibitem{ananev012}
	{\it Ананьевский, И. М.} Два подхода к управлению механической системой с неизвестными параметрами /
	И. М. Ананьевский // Изв. РАН. Теор. и сист. упр.— 2001.— № 2.— С. 39–47.
	
	\bibitem{anap80}
	{\it Анапольский, Л. Ю.} Метод сравнения в динамике дискретных систем /
	Л. Ю. Анапольский; ред. В. М. Матросов, Л. Ю. Анапольский // Вектор-функции Ляпунова и их построение.— Новосибирск: Наука, 1980.— С. 92–128.
	
	\bibitem{andr15}
	{\it Андреев, А. С.} Об управлении движением колесного мобильного робота /
	А. С. Андреев, О. А. Перегудова // ПММ.— 2015.— Т. 79.— № 4.— С. 451–462.
	
	\bibitem{andr971}
	{\it Андреев, А. С.} О стабилизации управляемых систем с гарантированной оценкой качества управления /
	А. С. Андреев, С. П. Безгласный // ПММ.— 1997.— Т. 61.— № 1.— С. 44–51.
	
	\bibitem{andr124}
	{\it Андреев, А. С.} Метод векторной функции Ляпунова в задаче об управлении систем
	с мгновенной обратной связью / А. С. Андреев, А. О. Артемова // Ученые записки
	Ульяновского государственного университета.— 2012.— № 1(4).— С. 15–19.
	
	\bibitem{andr126}
	{\it Андреев, А. С.} Об управлении движением голономной механической системы / А. С. Андреев,
	А. О. Артемова // Научно-технический вестник Поволжья.— 2012.— № 6.— С. 80–87.
	
	\bibitem{andr0232}
	{\it Андреев, А. С.} Знакопостоянные функции Ляпунова в задачах об устойчивости /
	А. С. Андреев, Т. А. Бойкова // Механика твердого тела.— 2002.— № 32.— С. 109–116.
	
	\bibitem{andr055}
	{\it Андреев, А. С.} К методу сравнения в задачах об асимптотической устойчивости /
	А. С. Андреев, О. А. Перегудова // Доклады Академии наук.— 2005.— Т. 400, № 5.—
	С. 621–624.
	
	\bibitem{andr078}
	{\it Андреев, А. С.} О стабилизации движения нестационарной управляемой системы /
	А. С. Андреев, В. В. Румянцев // Автоматика и телемеханика.— 2007.— № 8.—
	С. 18–31.
	
	\bibitem{andr10}
	{\it Андреев, А. С.} Андреев А.С., Павликов С.В. Устойчивость функционально-дифферен-
	циальных уравнений. Ульяновск: УлГУ, 2010. 203 с.
	
	\bibitem{andr136}
	{\it Андреев, А. С.} О моделировании цифрового регулятора на основе прямого метода Ляпунова /
	А. С. Андреев, Е. А. Кудашова, О. А. Перегудова // Научно-технический вестник Поволжья.— 2013.— № 6.—
	С. 113–115.
	
	\bibitem{andr111}
	{\it Андреев, А. С.} Об устойчивости нулевого решения системы с разрывной правой частью /
	А. С. Андреев, О. Г. Дмитриева, Ю. В. Петровичева // Научно-технический вестник Поволжья.— 2011.— № 1.—
	С. 15–21.
	
	\bibitem{andr044}
	{\it Андреев, А. С.} Об устойчивости неустановившегося решения механической системы /
	А. С. Андреев, Т. А. Бойкова // ПММ.— 2004.— Т. 68.— № 4.— С. 678–686.
	
	\bibitem{andr071}
	{\it Андреев, А. С.} Об устойчивости обощенного стационарного движения механической системы в зависимости от действующих сил /
	А. С. Андреев, Р. Б. Зайнетдинов // Труды IX Международной Четаевской Конференции "Аналитическая механика, устойчивость и управление движением", посвященной 105-летию Н. Г. Четаева.— Иркутск: Сибирское отделение РАН.— 2007.— Т. 1.— С. 5–14.
	
	\bibitem{andr141}
	{\it Андреев, А. С.} Вектор-функции Ляпунова в задачах о стабилизации движениц управляемых систем /
	А. С. Андреев, О. А. Перегудова // Журнал Средневолжского математического общества.— 2014.— Т. 16, № 1.—
	С. 32–44.
	
	\bibitem{andr14vspu}
	{\it Андреев, А. С.} О стабилизации программных движений голономной механической системы /
	А. С. Андреев, О. А. Перегудова // XII Всероссийское совещание по проблемам управления ВСПУ-2014. Институт проблем управления им. В. А. Трапезникова РАН.— 2014.— С. 1840–1843.
	
	\bibitem{kudash136}
	{\it Андреев А. С.} О моделировании цифрового регулятора на основе прямого метода Ляпунова / А. С. Андреев, О. А. Перегудова, Е. А. Кудашова //
	Научно-технический вестник Поволжья №6 2013. Казань: Научно-технический вестник Поволжья, 2013 - с. 113-115.
	
	\bibitem{kudash145}
	{\it Андреев А. С.} Синтез непрерывного и кусочно-постоянного управления движением колесного мобильного робота / А. С. Андреев, С. Ю. Раков, Е. А. Кудашова  //
	Научно-технический вестник Поволжья №5 2014. Казань: Научно-технический вестник Поволжья, 2014 — с.97-101


	\bibitem{kudash152}
	{\it Андреев А. С.} О моделировании структуры управления для колес-ного робота с омни-колесами / Андреев А. С., Е. А. Кудашова // Автоматизация процессов управления №2 (40) 2015. Ульяновск: Автоматизация процессов управления, 2015 — с.114-121
	
	
	\bibitem{kudash14ek}
	{\it Андреев А. С.} О стабилизации движений механических систем управлениями различного типа / А.С. Андреев, Е.А. Кудашова, О.А. Перегудова // Тезисы докладов Международной конференции, посвященной 90-летию со дня рождения академика Н.Н. Красовского «Динамика систем и про-цессы управления», 15-20 сентября 2014г., Екатеринбург. – с. 35-37

	
	\bibitem{artyomova126}
	{\it Артемова, А. О.} Моделирование управляемого движения двузвенного манипулятора на подвижном основании /
	А. О. Артемова // Научно-технический вестник Поволжья.— 2012.— № 6.
	
	\bibitem{artyomova135}
	{\it Артемова, А. О.} Об управлении пространственным движением многозвенного манипулятора на подвижном основании /
	А. О. Артемова, Е. Э. Звягинцева, Е. А. Кудашова // Научно-технический вестник Поволжья.— 2013.— № 5.— С. 106–109.
	
	\bibitem{afan89}
	{\it Афанасьев, В. Н.} Математическая теория конструирования систем управления /
	В. Н. Афанасьев, В. Б. Колмановский, В. Р. Носов.— М.: Высшая школа, 1989.— 447 с.

	%\bibitem{bogd084}
	\bibitem{barab10}
	{\it Барабанов, И. Н.} Динамические модели информационного управления в социальных сетях / И. Н. Барабанов, Н. А. Коргин, Д. А. Новиков, А. Г. Чхартишвили // “Динамические модели информационного управления в социальных сетях”, Автомат. и телемех., 2010, № 11, 172–182
	
	
	\bibitem{barkin08}
	{\it Баркин, А. И.} Об абсолютной устойчивости дискретных систем / А. И. Баркин // Автоматика и телемеханика.— 1998.— № 10.— С. 3–8.
	
	\bibitem{bellman67}
	{\it Беллман, Р.} Дифференциально-разностные уравнения / Р. Беллман, К. Кук.— М.: Мир,
	1967.— 548 с.
	
	\bibitem{belousov02}
	{\it Белоусов И.Р.} Белоусов И.Р. Формирование уравнений динамики роботов-  манипуляторов - М.: ИПМ им. М.В.Келдыша РАН, 2002
	
	\bibitem{blumin032}
	{\it Блюмин С. Л.} Дискретные математические модели Вольтерра в экологии и других областях /
	С. Л. Блюмин, А. М. Шмырин // Экология ЦЧО РФ.— 2003.— № 2.—
	С. 16–18.
	
	\bibitem{blumin044}
	{\it Блюмин С. Л.} Нечеткие системы Вольтерра /
	С. Л. Блюмин, А. М. Шмырин // Проблемы управления.— 2004.— № 4.—
	С. 75–58.
	
	%\bibitem{bogd598}
	\bibitem{barab95}
	{\it Барабанов, И. Н.} Построение функций Ляпунова для дискретных систем со случайными параметрами / И. Н. Барабанов // Автомат. и телемех., 1995, № 11, 31–41.
	
	\bibitem{bogd1404}
	{\it Богданов, А. Ю.} Об устойчивости точки покоя дискретной системы / А. Ю. Богданов, С. В. Черников // Ученые записки УлГУ. Сер. "Фундаментальные проблемы математики и механаники".— Ульяновск: УлГУ, 2004.— Вып. 1(14).- С. 99–115.
	

	\bibitem{kudash071}
	{\it Богданов А. Ю.} Устойчивость неавтономных дискретных систем типа Лотки-Вольтерра / А. Ю. Богданов , Е. А. Кудашова //
	Ученые записки УлГУ. Сер. Математика и информационные технологии. Вып. 1(18) - Ульяновск: УлГУ, 2007. – С. 182-188.
	
	\bibitem{bogdanovmono}
	{\it Богданов А. Ю.} Дискретные динамические системы: проблемы устойчивости и управления / А. Ю. Богданов //
	Ульяновск: УлГТУ, 2008.- 262 с.
	
	\bibitem{kudash0916}
	{\it Богданов А. Ю.} К вопросу об оптимальной стабилизации дискретных управляемых систем / Ю. А. Матвеев, А. Ю. Богданов, Т. Е. Исаева, Е. А. Кудашова  //
	Обозрение прикладной и промышленной математики.  Москва: ТВП. – 2009. – Том 16. – Вып. 3. – С. 505-507.
	
	\bibitem{kudash0946}
	{\it Богданов А. Ю.} О равномерной асимптотической устойчивости решений дискретных уравнений с изменяющейся структурой / А. Ю. Богданов, Е. А. Кудашова //
	Труды Седьмой международной конференции "Математическое моделирование физических, экономических, технических, социальных систем и процессов", 2-5 февраля 2009 г., г. Ульяновск, Россия. -Ульяновск, 2009. – С. 46-47.
	
	\bibitem{kudash092}
	{\it Богданов А. Ю.} Развитие прямого метода Ляпунова и равномерная асимптотическая устойчивость решений дискретных уравнений с изменяющейся структурой / А. Ю. Богданов, Е. А. Кудашова //
	Обозрение прикладной и промышленной математики.  Москва: ТВП. – 2009. – Том 16. – Вып. 2. – С. 294-295.
	
	\bibitem{kudash1013}
	{\it Богданов А. Ю.} Численные методы синтеза управления в нестационарных дискретных системах / А. Ю. Богданов, Е. А. Кудашова //
	Ученые записки УлГУ. Сер. Математика и информационные технологии. –Вып. 1(3). – Ульяновск: УлГУ, 2010. – С. 9.
	
	\bibitem{kudash1034}
	{\it Богданов А. Ю.} Стабилизация нестационарных дискретных систем на основе свойств диссипативности и пассивности / А. Ю. Богданов, Е. А. Кудашова //
	Труды всероссийского семинара «Аналитическая механика, устойчивость и управление движением», 15-18 июня 2010 г., г. Ульяновск. –Ульяновск:УлГУ, 2010. С. 34-37.
	
	\bibitem{kudash119}
	{\it Богданов А. Ю.} Необходимые и достаточные условия диссипативности и беспотерьности для одного класса нестационарных нелинейных управляемых систем / А. Ю. Богданов, Е. А. Кудашова //
	Труды всероссийского семинара «Аналитическая механика, устойчивость и управление движением», 9-12 июня 2011 г., г. Ульяновск. –Ульяновск:УлГУ, 2011. С. 54-57.
	
	
	\bibitem{borcov86}
	{\it Борцов, Ю. А.} Автоматические системы с разрывным управлением / Ю. А. Борцов,
	И. Б. Юнгер.— Л.: Энергоатомиздат. Ленингр. отделение, 1986.— 168 с.
	
	\bibitem{bromberg53}
	{\it Бромберг, П. В.} Устойчивость и автоколебания импульсных систем регулирования /
	П. В. Бромберг.— М.: Оборонгиз, 1953.— 224 с.
	
	\bibitem{bromberg67}
	{\it Бромберг, П. В.} Матричные методы в теории релейного и импульсного управления /
	П. В. Бромберг.— М.: Наука, 1967.— 324 с.
	
	\bibitem{bulg84}
	{\it Булгаков, Н. Г.} Знакопостоянные функции в теории устойчивости /
	Н. Г. Булгаков.— Минск: Университетское, 1984.— 78 с.
	
	\bibitem{bunich02}
	{\it Бунич, А. Л.} Синтез и применение дискретных систем управления с идентификатором /
	А. Л. Бунич, Н. Н. Бухтадзе // .: Наука, 2002
	
	\bibitem{burkov98}
	{\it Бурков И.В} Бурков И.В. Стабилизация натуральной механической системы без измерения ее скоростей с приложением к управлению твердым телом // Прикладная математика и механика. 1998. Т. 62, Вып. 6. С. 923-933.
	
	\bibitem{vasil819}
	{\it Васильев, С. Н.} Метод сравнения в анализе систем 1, 2 / С. Н. Васильев // Дифференц.
	уравнения.— 1981.— Т. 17, № 9.— С. 1562–1573.
	
	\bibitem{vidal74}
	{\it Видаль, П} Нелинейные импульсные системы / П. Видаль.— М.: Энергия, 1974.— 336 с.
	
	\bibitem{volterra76}
	{\it Вольтерра В.} Математическая теория борьбы за существование /
	В. Вольтерра.— М.: Наука, 1976.— 286 с.
	
	\bibitem{vukobr85}
	{\it Вуковбратович, М. К.} Управление манипуляционными роботами: Теория и применение /
	М. К. Вуковбратович, Д. М. Стокич // М.: Наука, 1985.— 384 с.
	
	\bibitem{vukobr82}
	{\it Вуковбратович, М. К.} Синтез управления возмущенным движением автоматических манипуляторов /
	М. К. Вуковбратович, Д. М. Стокич // Машиноведение.— 1982.— № 1.
	
	\bibitem{gajshun1297}
	{\it Гайшун И. В.} Дискретные уравнения с изменяющейся структурой и устойчивость их решений /
	И. В. Гайшун // Дифференциальные уравнения.— 1997.— Т. 33, № 12.— С. 1607–1614.
	
	\bibitem{gajshun697}
	{\it Гайшун И. В.} Устойчивость дискретных процессов Вольтерра с убывающим последействием /
	И. В. Гайшун // Автомотика и телемеханика.— 1997.— № 6.— С. 118–124.
	
	\bibitem{gajshun007}
	{\it Гайшун И. В.} Управляемость система, описываемых линейными дискретными уравнениями Вольтерра /
	И. В. Гайшун, М. П. Дымков // Автомотика и телемеханика.— 2000.— № 7.— С. 88–100.
	
	\bibitem{gajshun11}
	{\it Гайшун И. В.} Системы с дискретным временем /
	И. В. Гайшун.— М.: Минск, 2011.
	
	
	\bibitem{gajshun01}
	{\it Гайшун И. В.} Системы с дискретным временем /
	И. В. Гайшун.— Минск: Институт математики НАН Беларуси, 2001.
	
	\bibitem{gelig82}
	{\it Гелиг А. Х.} Динамика импульсных системы и нейронных сетей /
	А. Х. Гелиг.— Л.: Изд-во Ленингр. ан-та, 1982.
	
	\bibitem{gelig03}
	{\it Гелиг А. Х.} Устойчивость нелинейных импульсных систем по первому приближению /
	А. Х. Гелиг // ПММ.— 2003.— Т. 62, № 8.— С. 231–238.
	
	\bibitem{gelig04}
	{\it Гелиг А. Х.} Стабилизация нестационарных импульсных системы /
	А. Х. Гелиг, И. Е. Зубер // Автоматика и телемеханика.— 2004.— № 5.— С. 29–37.
	
	\bibitem{gelig78}
	{\it Гелиг А. Х.} Устойчивость нелинейных систем с неединственным состоянием равновесия /
	А. Х. Гелиг, Г. А. Леонов, В. А. Якубович // М.: Наука, 1978.
	
	\bibitem{gelig05}
	{\it Гелиг А. Х.} Стабилизируемость двух классов нелинейных импульсных систем с последействием /
	А. Х. Гелиг, В. А. Муранов // Вестник С.-Петербург. ун-та, Сер. 1.— 2005.— Вып. 3.— С. 3–15.
	
	\bibitem{gelig93}
	{\it Гелиг А. Х.} Колебания и устойчивость нелинейных импульсных систем /
	А. Х. Гелиг, А. Н. Чурилов // СПб.: Изд-во СПб ун-та, 1993.
	
	\bibitem{golubev05}
	{\it Голубев А. Е.} Стабилизация нелинейных динамических систем с использованием оценки состояния системы асимптотически наблюдателем / А. Е. Голубев, А. П. Крищенко, С. Б. Ткачев // Автоматика и телемеханика.— 2005.— № 7.— С. 3–42.
	
	
	%\bibitem{dobr95}
	%{\it Добрынина И. С.} Моделирование динамики манипуляционных роботов с применением метода декомпозиции управления / И. С. Добрынина // Изв. РАН. Техн. и кибернет.— 1995.— № 4.— С. 246–256.
	%\bibitem{dobr94}
	%{\it Добрынина И. С.} Компьютерное моделирование управлением движения системы связных тел / И. С. Добрынина, И. И. Карпов, Ф. Л. Черноусько  // Изв. РАН. Техн. и кибернет.— 1994.— № 1.— С. 167–180.
	%\bibitem{dorf04}
	%{\it Дорф Р.} Современные системы управления /
	%Р. Дорф, Р. Бишоп; Пер. с англ. Б. И. Копылова.— М.: Лаборатория Базовых знаний, 2004.— 832 с.
	%\bibitem{druzh074}
	%{\it Дружинин, Э. И.} Об устойчивости прямых алгоритмов расчета программных управлений
	%в нелинейных системах / Э. И. Дружинин // Известия РАН. Теория и системы
	%управления.— 2007.— Т. 3, № 4.— С. 14–20.
	%\bibitem{dixta03}
	%{\it Дыхта, В. А.} Оптимальное импульсное управление с приложениями / В. А. Дыхта, О. Н. Самсонюк // М.:
	%Физматлит, 2003.
	%\bibitem{emel06}
	%{\it Емельянов, С. В.} Избранные труды по теории управления / С. В. Емельянов.— М.:
	%Наука, 2006.— 450 с.
	%\bibitem{zavriev90}
	%{\it Завриев С. К.} Прямой метод Ляпунова в исследовании притяжения траекторий конечно-разностных включений / С. К. Завриев,  А. Г. Перевозчиков // Журнал вычислительной математики и математической физики.— 1990.— Т. 30, № 1.— С. 22–32.
	
	\bibitem{kudash131}
	{\it Звягинцева Е. Э.} Об управлении механической системой с циклическими координатами / Е. Э. Звягинцева, Е. А. Кудашова //
	Научно-технический вестник Поволжья №1 2013. Казань: Научно-технический вестник Поволжья, 2013 - с. 217-221.
	
	
	\bibitem{zobova06}
	{\it Зобова А. А.} Математические аспекты динамики движения экипажа с тремя окольцованными колесами / А. А. Зобова, Я. В. Татаринов //
	Сб. Мобильные роботы и мехатронные системы. М: Изд-во МГУ, 2006.- с. 61-67.
	
	\bibitem{zobova061}
	{\it Зобова А. А.} Свободное и управляемое движение некоторой модели экипажа с роликонесущими колесами / А. А. Зобова, Я. В. Татаринов //
	Вестник МГУ. Сер. 1. Математика, механика. 2008. №6. - с. 62-66.
	
	\bibitem{zobova09}
	{\it Зобова А. А.} Динамика экипажа с роликонесущими колесами / А. А. Зобова, Я. В. Татаринов //
	ПММ. 2009. Т. 73.- Вып. 1.- с. 13-22.
	
	\bibitem{kalen05}
	{\it Каленова В.И.} Неголономные механические системы и стабилизация движений / Каленова В.И., Карапетян A.B., Морозов В.М., Салмина М.А. // Фундаментальная и прикладная математика. — 2005. — Т. 11, вып. 7. — С. 117-158.
	
	\bibitem{kalman71}
	{\it Калман, Р.} Очерки по математической теории систем / Р. Калман, П. Фалб, М. Арбиб.—
	М.: Мир, 1971.— 400 с.
	
	
	\bibitem{kamaeva09}
	{\it Камаева, Р. А.} К задаче слежения для колесного мобильного робота с неизвестной матрицей инерции / Р. А. Камаева, О. А. Перегудова // Обозрение прикладной и промышленной математики. 2009.- Т.16.- Вып. 4.- с. 664-665.
	
	\bibitem{karap98}
	{\it Карапетян, А. В.} Устойчивость стационарных движений / А. В. Карапетян.—
	М.: УРСС, 1998.
	
	\bibitem{karap78}
	{\it Карапетян A.B.} Об устойчивости стационарных движений неголономных систем Чаплыгина // ПММ. 1978. - Т. 43, вып. 5. - С. 801-807.
	
	\bibitem{karap80}
	{\it Карапетян A.B.} К вопросу об устойчивости стационарных движений неголономных систем // ПММ. 1980. — Т. 44, вып. 3. — С. 418
	
	\bibitem{kirichenova196}
	{\it Кириченова О. В.} Метод функций Ляпунова для систем линейных разностных уравнения с почти периодическими коэффициентами / О. В. Кириченова, А. С. Котюргина, Р. К. Романовский //
	Сиб. мат. журн.ин.— 1996.— Т. 37, № 1.— С. 170–174.
	
	\bibitem{kirichenova198}
	{\it Кириченова О. В.} Об устойчивости решений нелинейных почти периодических систем разностных уравнений / О. В. Кириченова // Сиб. мат. журн.ин.— 1996.— Т. 39, № 1.— С. 45–48.
	
	\bibitem{kolmanov952}
	{\it Колмановский B. Б.} Об устойчивости некоторых дискретных процессов Вольтерра / В. Б. Колмановский, А. М. Родионов // Автоматика и телемеханника.— 1995, № 2.— С. 3–13.
	
	\bibitem{kolmanov9511}
	{\it Колмановский B. Б.} О применении второго метода Ляпунова к разностным уравнениям Вольтерра / В. Б. Колмановский // Автоматика и телемеханника.— 1995, № 11.— С. 50–64.
	
	\bibitem{kolmanov965}
	{\it Колмановский B. Б.} Устойчивостьь дискретных уравнений Вольтерра / В. Б. Колмановский // Доклады Академии наук.— 1996.— Т. 349, № 5.— С. 610–614.
	
	\bibitem{kra591}
	{\it Красовский, Н. Н.} Красовский Н.Н. Некоторые задачи теории устойчивости движения. М.:
	Физматгиз, 1959. 211 с.
	
	\bibitem{krasovsk63}
	{\it Красовский, Н. Н.} О стабилизации неустойчивых движений дополнительными силами при неполной обратной связи / Н. Н. Красовский,
	// Прикладная математика и механика.— 1963.— Т. XXVII, № 4.— С. 641–663.
	
	\bibitem{krasovsk664}
	{\it Красовский, Н. Н.} Проблемы стабилизации управляемых движений / Н. Н. Красовский
	// Малкин, И. Г. Теория устойчивости движения. Доп. 4 / И. Г. Малкин.— М.:
	Наука, 1966.— С. 475–514.
	
	\bibitem{krutko873}
	{\it Крутько, П. Д.} Метод обратных задач динамики в теории конструирования алгоритмов
	управления манипуляционных роботов. задача стабилизации / П. Д. Крутько, Н. А. Лакота
	// Изв. АН СССР. Техническая кибернетика.— 1987.— № 3.— С. 23–30.
	
	\bibitem{kudash12}
	{\it Кудашова, Е. А.} Задача об управлении механическими системами. Синтез непрерывного и кусочно-непрерывного стабилизируеющего управления / Е. А. Кудашова // Ученые записки УлГУ. Сер. "Математика и информационные технологии".— Вып. 1. - Ульяновск: УлГУ, 2012.— С. 23–30.
	
	\bibitem{kudash0967}
	{\it Кудашова Е. А.} Об асимптотическом поведении решений неавтономной нелинейной системы второго порядка / Е. А. Кудашова //
	Труды Симбирской молодежной научной школы по аналитической динамике, устойчивости и управлению движениями и процессами, 8-12 июня 2009 г., г. Ульяновск. –Ульяновск:УлГУ, 2009. С. 67-68.
	
	\bibitem{kudash122}
	{\it Кудашова Е. А.} Прямой метод Ляпунова в задаче об устойчивости неавтономных дискретных систем типа Лотки - Вольтерра / Е. А. Кудашова //
	Труды Х международной Четаевской конференции «Аналитическая механика, устойчивость и управление», 12-16 июня 2012г., г. Казань. – Том 2. – Сек. 2. Устойчивость. – Казань: КНИТУ КАИ. – с. 316-322.
	
	\bibitem{kudash151}
	{\it Кудашова Е. А.} Метод векторных функций Ляпунова в задаче об асимптотической устойчивости разностных систем / О. А. Перегудова, Е. А. Кудашова //
	Научно-технический вестник Поволжья №1 2015. Казань: Научно-технический вестник Поволжья, 2015.— с.118-121


\bibitem{kudash14su}
{\it Кудашова Е. А.} О стабилизации механической системы с одной степенью свободы и с цифровым управлением // Труды международной конференции по математической теории управ-ления и механике, 3-7 июля 2015г., г. Суздаль. – с. 80-81

\bibitem{kudashpat}
{\it Кудашова Е. А.} Стабилизация движения трехколесного робота //  Патент РФ на программу для ЭВМ №2015615314. Москва, Роспатент, 2015. 

	
	\bibitem{kuleshov71}
	{\it Кулешов, В. С.} Динамика систем управления манипуляторами / В. С. Кулешов,
	Н. А. Лакота.— М.: Энергия, 1971.— 304 с.
	
	\bibitem{peregudova11}
	{\it Кузьмин, А. В.} Программная реализация алгоритма построения управления мобильным колесным роботом при учете проскальзывания колес / О. А. Перегудова, А. В. Кузьмин, Д. Ю. Моторина // Автоматика и телемеханика.—  2011.— № 4.
	
	\bibitem{kuznecov05}
	{\it Кузнецов, Н. В.} Критерии неустойчивости по первому приближению нелинейных дискретных систем / Н. В. Кузнецов, Г. А. Леонов // Вестник С.-Петерб. ун-та, Сер. 1.—  2005.— Вып. 3. С. 30-42.
	
	\bibitem{kunzevi08}
	{\it Кунцевич, В. М.} Синтез робастно устойчивых дискретных систем управления нелинейными объектами
	/ В. М. Кунцевич, А. А. Кунцевич // Автоматика и телемеханика.— 2008.— № 12.— С. 105-118.
	
	\bibitem{kunzevi77}
	{\it Кунцевич, В. М.} Синтез систем автоматического управления с помощью функций
	Ляпунова / В. М. Кунцевич, М. М. Лычак.— М.: Наука, 1977.— 400 с.
	
	\bibitem{lefsec67}
	{\it Лефшец С.} Устойчивость нелинейных систем автоматического управления /
	С. Лефшец. — М.: Мир, 1967.
	
	\bibitem{lakshm91}
	{\it Лакшмикантам В.} Устойчивость движения: метод сравнения / В. Лакшмикантам, С. Лила, А. А. Мартынюк. — Киев: Наукова думка, 1991.— 248 с.
	
	\bibitem{leonov02}
	{\it Леонов, Г. А.} Проблема Броккета для линейных дискретных систем управления / Г. А. Леонов // Автоматика и телемеханика.—  2002.— № 5. С. 92-96.
	
	\bibitem{lyapunov50}
	{\it Ляпунов А. М.} Общая задача об устойчивости движения /
	А. М. Ляпунов. — М.: Гостехиздат, 1950.
	
	\bibitem{malikov98}
	{\it Маликов, А. И.} Вектор-функции Ляпунова в анализе свойств систем со структурными изменениями / А. И. Маликов, В. М. Матросов // Дифференц. уравнения.— 1998.— № 2.— С. 47–54.
	530 с.
	
	\bibitem{malkin66}
	{\it Малкин, И. Г.} Теория устойчивости движения / И. Г. Малкин.— М.: Наука, 1966.—
	530 с.
	
	\bibitem{markeev99}
	{\it Маркеев, А. П.} Теоретическая механика / А. П. Маркеев.— М.: ЧеРо, 1999.— 569 с.
	
	\bibitem{martynenko07}
	{\it Мартыненко, Ю. Г.} О движении мобильного робота с роликонесущими колесами / Ю. Г. Мартыненко, А. М. Формальский  //
	Известия РАН. Теория и системы управления.— 2007.— № 6.— С. 142–149.
	
	
	\bibitem{martynenko10}
	{\it Мартыненко, Ю. Г.} Устойчивость стационарных движений мобильного робота с роликонесущими колесами и смещенным центром масс / Ю. Г. Мартыненко //
	ПММ.— 2010.— Т. 74.— Вып. 4.— С. 610–619.
	
	\bibitem{martynenko05}
	{\it Мартыненко Ю.Г.} Управление движением мобильных колесных роботов. // Фундаментальная и прикладная математика. — 2005. — Т. 11, вып. 8. — С. 29-80.
	
	
	\bibitem{martynuk00}
	{\it Мартынюк, А. А.} Анализ устойчивости дискретных систем / А. А. Мартынюк
	// Прикладная механика.— 2000.— Т. 36, № 7.— С. 3–35.
	
	\bibitem{matrosov68}
	{\it Матросов, В. М.} Принцип сравнения с вектор-функцией Ляпунова 1, 2 / В. М. Матросов
	// Дифференц. уравнения.— 1968.— Т. 4, № 8.— С. 1374–1386.
	
	\bibitem{matrosov01}
	{\it Матросов, В. М.} Метод векторных функций Ляпунова: анализ динамических свойств нелинейных систем / В. М. Матросов.— М.: Физматлит, 2001.— 380 с.
	
	\bibitem{matuhun899}
	{\it Матюхин, В. И.} Управление движением манипуляционных роботов на принципе декомпозиции при учете динамики приводов
	/ В. И. Матюхин, Е. С. Пятницкий // Автоматика и телемеханика.— 1989.— № 9.— С. 67–72.
	
	\bibitem{matuhun893}
	{\it Матюхин, В. И.} Устойчивость движений манипуляционных роботов в режиме декомпозиции
	/ В. И. Матюхин // Автоматика и телемеханика.— 1989.— № 3.— С. 33–44.
	
	\bibitem{matuhun9311}
	{\it Матюхин, В. И.} Устойчивость движения механических систем при учете постоянно действующих возмущений
	/ В. И. Матюхин // Автоматика и телемеханика.— 1993.— № 11.— С. 124–134.
	
	\bibitem{matuhun961}
	{\it Матюхин, В. И.} Сильная устойчивость движений механических систем
	/ В. И. Матюхин // Автоматика и телемеханика.— 1996.— № 1.— С. 37–56.
	
	\bibitem{matuhun01}
	{\it Матюхин, В. И.} Универсальные законы управления механическими системами /
	В. И. Матюхин.— М.: МАКС Пресс, 2001.— 252 с.
	
	\bibitem{matuhun048}
	{\it Матюхин, В. И.} Управляемость механических систем в классе управлений, ограниченных вместе с производной
	/ В. И. Матюхин, Е. С. Пятницкий // Автоматика и телемеханика.— 2004.— № 8.— С. 14–38.
	
	\bibitem{matuhun09}
	{\it Матюхин, В. И.} Управление механическими системами / В. И. Матюхин.— М.: Физматлит,
	2009.— 320 с.
	
	\bibitem{matuhun10}
	{\it Матюхин, В. И.} Управление движением манипулятора / В. И. Матюхин.— М.: ИПУ
	РАН, 2010.— 96 с.
	
	\bibitem{matuhun88}
	{\it Михеев, Ю. В.} Ассимтотический анализ цифровых систем управления / Ю. В. Михеев, В. А. Соболев, Э. М. Фридман // Автоматика и телемеханика.— 1988.— № 5.— С. 83–88.
	
	\bibitem{medvedev78}
	{\it Медведев, В. С.} Системы управления манипуляционных роботов / В. С. Медведев,
	А. Г. Лесков, А. С. Ющенко.— М.: Наука, 1978.— 416 с.
	
	\bibitem{gudova10}
	{\it Моторина, Д. Ю.} Об отслеживании траектории колесного робота с неизвестной массой с помощью непрерывного управления с запаздыванием / О. А. Перегудова, Д. Ю. Моторина // Материалы конференции "Управление в технических системах" (УТС-2010). Санкт-Петербург: ОАО "Концерн "ЦНИИ Электроприбор"".—  2010.— С. 362-365.
	
	\bibitem{motorina11}
	{\it Моторина, Д. Ю.} Алгоритм построения запаздывающего управления для мобильного робота при учете эффекта проскальзывания колес / Д. Ю. Моторина // Обозрение прикладной и промышленной математики.—  2010.— Т.— 17.— Вып. 5.— С. 753-754.
	
	\bibitem{motorina10}
	{\it Моторина, Д. Ю.} Синтез управления для механических систем с неизвестной матрицей инерции при учете запаздывания в структуре обратной связи / Д. Ю. Моторина // Автоматизация процессов управления.—  2010.— №. 4.
	
	\bibitem{motorina101}
	{\it Моторина Д.Ю.} Построение алгоритма синтеза управления с насыщением в задаче слежения для колесного мобильного робота // Журнал Средневолжского математического общества. — 2010. — Т. 12, №3. -С. 102-110.
	
	%\bibitem{mishkis67}
	%{\it Мышкис, А. Д.} Система с толчками в заданные моменты времени / А. Д. Мышкис
	%// Математ. сборник.— 1967.— Т. 74 (116), № 2.— С. 202–208.
	
	\bibitem{pahomov132}
	{\it Пахомов, К. В.} Синтез запаздывающего управления движением колесного робота на основе метода бэкстеппинга /
	К. В. Пахомов, О. А. Перегудова, Е. В. Филаткина // Научно-технический вестник Поволжья.— 2013.— № 2.— С. 37–40.
	
	\bibitem{peregudova09}
	{\it Перегудова, О. А.} Метод сравнения в задачах устойчивости и управления движениями
	механических систем / О. А. Перегудова.— Ульяновск: Изд-во УлГУ, 2009.— 253 с.
	
	\bibitem{peregudova092}
	{\it Перегудова, О. А.} О стабилизации движений неавтономных механических систем / О. А. Перегудова // ПММ.—  2009.— Т. 72.— Вып. 4.— С. 620.
	
	\bibitem{peregudova07}
	{\it Перегудова, О. А.} Уравнения сравнения в задачах об устойчивости движения / О. А. Перегудова // Автоматика и телемеханика.—  2007.— № 9.— С. 56–63.
	
	\bibitem{peregudova13}
	{\it Перегудова, О. А.} О стабилизации нелинейных систем с кусочно-постоянным управлением при помощи метода бэкстеппинга / О. А. Перегудова, К. В. Пахомов // Автоматизация процессов управления.—  2013.— № 4(34).
	
	\bibitem{peregudova14}
	{\it Перегудова, О. А.} Перегудова О.А., Макаров Д.С.  Синтез управления двухзвенным манипулятором // Автоматизация процессов управления. – 2014. – № 4(38). – С. 36–41.
	
	\bibitem{petrov79}
	{\it Петров, Б. Н.} Построение алгоритмов управления как обратная задача динамики / Б. Н. Петров, П. Д. Крутько, Е. П. Попов // Докл. АН СССР.— 1979.— № 5.— С. 1078–1081.
	
	\bibitem{popov78}
	{\it Попов, Е. П.} Системы управления манипуляционных роботов / Е. П. Попов, А. Ф. Верещагин,
	С. Л. Зенкевич.— М.: Наука, 1978.— 400 с.
	
	\bibitem{popov781}
	{\it Попов, Е. П.} Манипуляционные роботы. Динамика и алгоритмы. / Е. П. Попов, А. Ф. Верещагин,
	С. Л. Зенкевич.— М.: Наука, 1978.— 398 с.
	
	\bibitem{pjatnic873}
	{\it Пятницкий, Е. С.} Синтез управления манипуляционными роботами на принципе декомпозиции
	/ Е. С. Пятницкий // Известия АН СССР. Техническая кибернетика.—
	1987.— № 3.— С. 92–99.
	
	\bibitem{pjatnic882}
	{\it Пятницкий, Е. С.} Принцип декомпозиции и в управлении механическими системами
	/ Е. С. Пятницкий // ДАН СССР.— 1988.— Т.— 300.— № 2.— С. 300-303.
	
	\bibitem{pjatnic937}
	{\it Пятницкий, Е. С.} Синтез систем стабилизации программных движений нелинейных объектов управления
	/ Е. С. Пятницкий // Автоматика и телемеханика.— 1993.— № 7.— С. 19-37.
	
	\bibitem{remshin97}
	{\it Ремшин, С. А.} Синтез управления двузвенным манипулятором /
	С. А. Ремшин // Известия РАН. Теор. и сист. упр.— 1997.— № 2.— С. 146–150.
	
	\bibitem{remshin98}
	{\it Ремшин, С. А.} Синтез управления в нелинейной динамической систему на основе декомпозиции /
	С. А. Ремшин, Ф. Л. Черноусько // Прикл. матем. и мех. — 1998.— Т. 62, вып. 1.— С. 121–128.
	
	%\bibitem{rozer77}
	%{\it Розенвассер, Е. Н.} Показатели Ляпунова в теории линейных систем управления / Е. Н. Розенвассер.— М.: Наука, 1977.
	
	\bibitem{rodionov00}
	{\it Родионов, А. М.} Притяжение для дискретных уравнений, приложение к динамике популяций
	/ А. М. Родионов // Автоматика и телемеханика.— 2000.— № 2.— С. 76-85.
	
	\bibitem{rodionov0012}
	{\it Родионов, А. М.} О некоторых дискретных моделях межвидового взаимодействия
	/ А. М. Родионов // Автоматика и телемеханика.— 2000.— № 12.— С. 122-129.
	
	\bibitem{rojtenberg71}
	{\it Ройтенберг Я. Н.} Автоматическое управление /
	Я. Н. Ройтенберг. — М.: Наука, 1971.— 395 с.
	
	\bibitem{rumyncev87}
	{\it Румянцев, В. В.} Устойчивость и стабилизация движения по отношению к части переменных /
	В. В. Румянцев, А. С. Озиранер. — М.: Наука, 1987,- 253 с.
	
	\bibitem{samarsky73}
	{\it Самарский, А. А.} Устойчивость разностных схем /
	А. А. Самарский, А. В. Гулин. — М.: Наука, 1973,- 397 с.
	
	\bibitem{samoj87}
	{\it Самойленко, А. М.} Дифференциальные уравнения с импульсным воздействием /
	А. М. Самойленко. — Киев.: Вища школа, 1987.
	
	\bibitem{timof96}
	{\it Тимофеев А. В.} Устойчивость и стабилизация программного движения робота-манипулятора /
	А. В. Тимофеев, Ю. В. Экало // Автоматика и телемеханика.— 1996.— № 10.
	
	\bibitem{halanaj71}
	{\it Халанай А.} Качественная теория импульсных систем /
	А. Халанай, Д. Векслер. — М.: Мир, 1971.- 309 с.
	
	\bibitem{halil09}
	{\it Халил Х. К.} Нелинейные системы / Х. К. Халил.— М.: Ижевск: НИЦ "Регулярная и хаотическая динамика", Институт клмпьютерных исследований, 2009.— 832 с.
	
	\bibitem{heil84}
	{\it Хейл Дж.} Хейл Дж. Теория функционально-дифферепнциальных уравнений. М.: Мир,
	1984. 421 с.
	
	\bibitem{utkin74}
	{\it Уткин, В. И.} Скользящие режими и их применения в системах с переменной структурой /
	В. И. Уткин. — М.: Наука, 1974.
	
	\bibitem{filip85}
	{\it Филиппов, А. Ф.} Дифференциальные уравнения с разрывной правой частью /
	А. Ф. Филиппов. — М.: Наука, 1985.
	
	\bibitem{finogenko07}
	{\it Финогенко, И. А.} О задачах слежения, управляемости и стабилизации для механических систем с использованием комбинаций разрывных обратных связей и импульсных управлений /
	И. А. Финогенко // Труды IX Международной Четаевской Конференции "Аналитическая механика, устойчивость и управление движением", посвященной 105-летию Н. Г. Четаева.— Иркутск: Сибирское отделение РАН.— 2007.— Т. 2.— С. 299–307.
	
	\bibitem{fish00}
	{\it Фишман, Л. З.} О сохранении областей притяжения при дискретизации непрерывных систем
	/ А. М. Родионов // Автоматика и телемеханика.— 2000.— № 5.— С. 93-97.
	
	\bibitem{furasov74}
	{\it Фурасов, В. Д.} Устойчивость и стабилизация дискретных процессов /
	В. Д. Фурасов. — М.: Наука, 1982.- 192 с.
	
	\bibitem{halfman72}
	{\it Халфман Р.} Халфман Р. Динамика. — М.: Наука, 1972. — 568 с.
	
	\bibitem{cipkin63}
	{\it Цыпкин, Я. З.} Теория линейных импульсных систем /
	Я. З. Цыпкин. — М.: Наука, 1963.- 968 с.
	
	\bibitem{cipkin73}
	{\it Цыпкин, Я. З.} Теория нелинейных импульсных систем /
	Я. З. Цыпкин, Ю. С. Попков. — М.: Наука, 1973.- 414 с.
	
	\bibitem{cipkin84}
	{\it Цыпкин, Я. З.} Дискретные адаптивные системы управления /
	Я. З. Цыпкин, Г. К. Кельманс // Итоги науки и техники ВИНИТИ. Серия "Техническая кибернетика". — М.: 1984.— № 17.
	
	\bibitem{chernou06}
	{\it Черноусько, Ф. Л.} Методы управления нелинейными механическими системами /
	Ф. Л. Черноусько, И. М. Ананьевский, С. А. Решмин.— М.: Физматлит, 2006.— 326 с.
	
	\bibitem{chernou89}
	{\it Черноусько, Ф. Л.} Манипуляционные роботы: динамика, управление, оптимизация /
	Ф. Л. Черноусько, Н. Н. Болотник, В. Г. Градецкий.— М.: Физматлит, 1989.— 368 с.
	
	\bibitem{churil00}
	{\it Чурилов А. Н.} Стабилизация линейной системы с помощью комбинированной импульсной модяции /
	А. Н. Чурилов // Автоматика и телемеханика.— 2000.— № 10. С. 71-76.
	
	\bibitem{shepelyavij70}
	{\it Шепелявый А. И.} О качественном исследовании устойчивости в целом и неустойчивости амплитудно-импульсных систем /
	А. И. Шепелявый // Доклады АН СССР.— 1970.— Т. 190.— № 5. С. 1044-1047.
	
	
	\bibitem{jurevic05}
	{\it Юревич, Е. И.} Основы робототехники / Е. И. Юревич.— 2-е изд.— СПб.:
	БХВ-Петербург, 2005.— 416 с.
	
	\bibitem{jurevic07}
	{\it Юревич, Е. И.} Теория автоматического управления / Е. И. Юревич.— 3-е изд.— СПб.:
	БХВ-Петербург, 2007.— 560 с.
	
	\bibitem{jurkov01}
	{\it Юрков, А. В.} Задачи стабилизации программных движений управляемых динамических систем / А. В. Юрков // Электронный журнал "Исследовано в России".— 2001.— С. 147-164.
	
	\bibitem{alonge03}
	{\it Alonge F.} Alonge F., D’Ippolito F., Raimondi F.M. An adaptive control law for robotic manipulator without velocity feedback // Control Engineering Practice. 2003. 11. P. 999-1005.
	
	\bibitem{araki71}
	{\it Araki, M.} Stability of sampled-data composite systems with many nonlinearities / M. Araki, K. Ando, B. Kondo // IEEE Trans. Automat. Contr.— AC-16, 1971.— Pp. 22–27.
	
	\bibitem{artstein2576}
	{\it Artstein, Z.} Limiting equations and stability of nonautonomous ordinary differential equations. In the Stability of Dynamical Systems, (Appendix A), CBMS Regional Conference Series in Applied Mathematics.— Vol. 25, SIAM, Philadelphia, 1976.— Pp. 57–76.
	
	\bibitem{artstein76}
	{\it Artstein, Z.} On the limiting equations and invariance of time-dependent difference equations // Stability of dynamical systems (Theory and applications) Proceedings of NSF conference, Mississippi State University .— 1976.— Pp. 3–9.
	
	\bibitem{artstein77}
	{\it Artstein, Z.} Topological dynamics of an ordinary differential equation / Z. Artstein // J.
	Different. Equat.— 1977.— Vol. 23, no. 2.— Pp. 216–223.
	
	\bibitem{basson97}
	{\it Basson, M.} Harvesting in discrete-time predator-prey systems / M. Basson, M. J. Fogarty // Math. Biosci.— 1977.— Vol. 141, no. 1.— Pp. 41–47.
	
	\bibitem{balafoutis}
	{\it Balafoutis C.} Balafoutis C., Patel R., Misra P. Efficient modeling and computation of manipulator dynamics using orthogonal cartesian tensors. IEEE J. of Rob. and Autom., 4, N 6, pp.665-676.
	
	\bibitem{barricelli}
	{\it Barricelli N.} Barricelli, Nils Aall (1954). «Esempi numerici di processi di evoluzione». Methodos: 45–68.
	
	\bibitem{berghuis91}
	{\it Berghuis H.} Berghuis H., Lѐohnberg P., Nijmeijer H. Tracking control of robots using only position measurements // 30th Conf. on Decision and Control. 1991. Vol. 1. P. 1039-1040.
	
	\bibitem{berghuis93_1}
	{\it Berghuis H.} Berghuis H. and Nijmeijer H. Global regulation of robots using only position measurements // Systems and Contr. Letters, vol. 21, 1993, pp. 289–293. 
	
	\bibitem{berghuis93_2}
	{\it Berghuis H.} Berghuis H. and Nijmeijer H. A passivity approach to controller-observer design for robots // IEEE Transactions on robotics and automation. 1993. Vol. 9. No 6. P. 740-754.
	
	\bibitem{burkov95}
	{\it Burkov I.V.} Burkov I.V. Stabilization of mechanical systems via bounded control and without velocity measurement //2nd Russian-Swedish Control Conf. St. Petersburg: St. Petersburg Technical Univ. 1995. P. 37-41.
	%\bibitem{d301}
	%{\it Bogdanov A. Yu.} The Discrete Nonlinear Nostationary Lossless Systems: Stabilization and Feedback Equivalence // Dynamic Systems and Applications.— Vol. 3.- Dynamic Publishers, Inc, USA.- 2001.— Pp. 91–98.
	
	\bibitem{calugi02}
	{\it Calugi F.} Calugi F., Robertsson A. and Johansson R. Output feedback adaptive control of robot manipulators using observer backstepping // Proceedings of the 2002 IEEE/RSJ Int. Conference of Intelligent Robots and Systems. Lausanne, Switzerland. 2002. P.2091-2096.
	
	\bibitem{choi04}
	{\it Choi S. K.} H-stability for nonlinear perturbated difference systems / S. K. Choi, N. J. Koo, S. M. Song // Bull. Korean Math. Soc. 41 (2004), No. 3.- Pp. 435-450.
	
	\bibitem{choi07}
	{\it Choi S. K.} Asymptotic behavior of nonlinear volterra difference systems / S. K. Choi, Y. H. Goo, N. J. Koo // Bull. Korean Math. Soc. 44 (2007), No. 1.- Pp. 177-184.
	
	
	\bibitem{corradini02}
	{\it Corradini M.} Experimental testing of a discrete-time sliding mode controller for trajectory tracking of a wheeled mobile robot in the presence of skidding effects / M. Corradini, T. Leo, G. Orlando // J. of Rob. Syst. 2002. V. 19.- Pp. 177-188.
	
	\bibitem{damoto01}
	{\it Damoto R.} Holonomic omni-directional vehicle with new omni-who mechanism / R. Damoto, W. Cheng, S. Hirose // Proc. of the 2001 IEEE Int. Conf. on Robotics and Automation. Seul, Korea, 2001.— Pp. 773–778.
	
	
	\bibitem{dash88}
	{\it Dash, A. T.} Polygamy in human and animal species / A. T. Dash, R. Gressman // Math. Biosci.— 1988.— Vol. 88, no. 1.— Pp. 49–66.
	
	\bibitem{dixon04}
	{\it Dixon W.E.} Dixon W.E., Zergeroglu E. and Dawson D.M. Global robust output feedback tracking control of robot manipulators // Robotica. 2004. Vol. 22. P. 352-357.
	
	\bibitem{nunes08}
	{\it Eduardo V. L. Nunes} Eduardo V. L. Nunes, Liu Hsu and Fernando Lizarralde. Arbitrarily small damping allows global output feedback tracking of a class of Euler-Lagrange systems // American Control Conference Westin Seattle Hotel, Seattle, Washington, USA, June 11-13, 2008. P. 377-382.
	
	\bibitem{elaydi04}
	{\it Elaydi S.} An introduction to Difference Equations. Third Edition. Springer-Verlad. - New York, 2004.
	
	\bibitem{elaydi89}
	{\it Elaydi S.} Stability of difference equations. Differencial equations and applications / S. Elaydi, A. Peterson // Proc. Int. Conf., Columbus/OH (USA).— 1989.— Pp. 235-238.
	
	\bibitem{fraser57}
	{\it Fraser A.} Fraser, Alex (1957). «Simulation of genetic systems by automatic digital computers. I. Introduction». Aust. J. Biol. Sci. 10: 484–491.
	
	\bibitem{geijtenbeek13}
	{\it Geijtenbeek T.} Geijtenbeek T., M. van de Panne, A. Frank van der Stappentitle Flexible Muscle-Based Locomotion for Bipedal Creatures. ACM Transactions on Graphics, ACM SIGGRAPH, vol 32, N 6, 2013
	
	\bibitem{hahn59}
	{\it Hahn W.} Theorie and Anwendung der direkten Methode von Lyapunov. Springer-Verlad. — Berlin, 1959.
	
	\bibitem{hsien88}
	{\it Hsien Y.} The phenomenon of unstable oscilation in population models / Y. Hsien // Math. Comput. Model.— 1988.— Vol. 10, no. 6.— Pp. 429–435.
	
	%\bibitem{isidori95}
	%{\it Isidori A.} Nonlinear Control Systems. 3rd ed.— Berlin: Springer, 1995.
	
	%\bibitem{isidori99}
	%{\it Isidori A.} Nonlinear Control Systems. II.— London: Springer, 1999.
	
	\bibitem{kelly93}
	{\it Kelly R.} Kelly R. A simple set-point robot controller by using only position measurements // In Preprint 12th IFAC World Congress, vol. 6, Sydney, 1993, pp. 173–176.
	
	\bibitem{kurzweil88}
	{\it Kurzweil J.} Structural stability of linear discrete systems via tthe exponential dichotomy / J. Kurzweil, G. Papaschinopoulos // Grech. Math. J.— 1988.— Vol. 38 (113), no. 2.— Pp. 280–284.
	
	\bibitem{lakshm90}
	{\it Lacshmikantham V.} Stability analysis of nonlinear systems / V. Lacshmikantham, S. Leela, A. A. Martynyuk // Singapore: World Scientific.— 1990.— 207 p.
	
	\bibitem{lakshm89}
	{\it Lacshmikantham V.} Stability analysis of nonlinear systems / V. Lacshmikantham, S. Leela, A. A. Martynyuk // New York: World Scientific.— 1989.— 207 p.
	
	\bibitem{lakshm02}
	{\it Lacshmikantham V.} Theory of difference equations: numerical methods and applications / V. Lacshmikantham, D. Trigiante // New York: Marcel Dekker, Inc.— 2002.— 320 p.
	
	\bibitem{lasalle76}
	{\it LaSalle J.P.} The stability of dynamical systems. SIAM, Philadelphia, Pennsilvania, 1976. – 76 p.
	
	\bibitem{lasalle77}
	{\it LaSalle J.P.} Stability of difference equations. In a Study in Ordinary Differential Equations (edited by J. K. Hale), Studies in Mathematical Series, Mathematical Association of America, 1977.
	
	\bibitem{lasalle86}
	{\it LaSalle J.P.} The stability and control of discrete processes. (Applied mathematical sciences; vol. 62), Springer-Verlag, 1986.- 147 p.
	
	\bibitem{lin2595}
	{\it Lin W.} Feedback stabilization of general nonlinear control systems: a passive systems approach // Systems and control letters.- 1995.- Vol. 25.- Pp. 41-52.
	
	\bibitem{lin25952}
	{\it Lin W.} Passivity and absolute stabilization of a class of discrete-time nonlinear systems / W. Lin, C. I. Byrnes // Automatica.- 1995.- Vol. 32(2).- Pp. 263-267.
	
	\bibitem{liu03}
	{\it Liu D.} Asymptotic stability of a class of linear discrete systems with multiple independent variables / D. Liu, A. Molchanov // Circuits systems signal processing. Vol. 22,. No. 3, 2003.- Pp. 307-324.
	
	\bibitem{liu07}
	{\it Liu Y.} Integrated control and navigation for omni-directional mobile robot based on trajectory linearization / Y. Liu, R. L. Williams II, J. J. Zhu // Proceedings of the 2007 American Control Conference. New York. USA. 2007.- Pp. 2153-2158.
	
	\bibitem{loria95}
	{\it Loria A.} Loria A., Ortega R. On tracking control of rigid and flexible joints robots // Appl. Math. Comput. Sci. 5(2), 1995, pp.101-113.
	
	\bibitem{loria96}
	{\it Loria A.} Loria A. Global tracking control of one degree of freedom Euler-Lagrange systems without velocity measurements // European J. Contr., vol. 2, 1996, pp. 144–151.
	
	\bibitem{loria98}
	{\it Loria A.} Loria A., Nijmeijer H. Bounded output feedback tracking control of fully actuated Euler–Lagrange systems // Systems and Control Letters. 1998. 33 (3). P. 151-161.
	
	\bibitem{mahil82}
	{\it Mahil S.} Mahil S. On the application of Lagrange's method to the description of dynamic systems. IEEE Trans. on SMC, vol SMC-12, N 6, 1982.
	
	\bibitem{mickens00}
	{\it Mickens R.} Applications of nonstandard finite differece schemes. World Scientific. Singapore, 2000.
	
	\bibitem{monaco97}
	{\it Monaco S.} On the conditions of passivity and losslessness in discrete time. / S. Monaco, D. Normand-Cyrot // Proc. European control conference.- 1997.- 5 p.
	
	\bibitem{navrol991}
	{\it Navarro-Lopez E. M.} Implications of dissipativity and passivity in the discrete-time setting. / E. M. Navarro-Lopez, D. Cortes, E. Fossas-Colet.- 1999.
	
	\bibitem{navrol992}
	{\it Navarro-Lopez E. M.} Dissipativity, passivity and feedback passivity in the nonlinear discrete-time setting. / E. M. Navarro-Lopez, E. Fossas-Colet.- 1999.
	
	\bibitem{nesic03}
	{\it Nesic D.} On uniform asymptotic stability of time-varying parameterized discrete-time cascades / D. Nesic, A. Loria // arXiv: math/0307167v1 [math.OC] 11 Jul 2003.
	
	\bibitem{nicosia90}
	{\it Nicosia S.} Nicosia S., Tomei P. Robot control by using only joint position measurements // IEEE Trans. Aut. Contr. 1990. V. 35, № 9. P. 1058-1061.
	
	\bibitem{nino06}
	{\it Nino-Suarez P. A.} Discrete-time feedback linearization of a wheeled mobile robot subject to transport delay / P. A.  Nino-Suarez, M. Velasco-Villa, E. Aranda-Bricaire // Congreso Latinoamericano de Control Automatico, La Habana Cuba, 2006.
	
	\bibitem{nino061}
	{\it Nino-Suarez P. A.} Discrete-time sliding mode path-tracking control for a wheeled mobile robot / P. A.  Nino-Suarez, M. Velasco-Villa, E. Aranda-Bricaire // 45th IEEE Conference on Decision and Control. San Diego, CA, USA, 2006. Pp. 3052-3057.
	
	\bibitem{oliveira08}
	{\it Oliveira H. P} Precise Modeling of a Four Wheeles Omni-directional Robot / H. P. Oliveira, A. J. Sousa, A. P. Moreira, P. J. Costa // Proceedings of the 8th Conference on Autonomous Robot Systems and Competitions, 2008.
	
	\bibitem{orosco04}
	{\it Orosco-Guerro G.} Discrete-time controller for a wheeled mobile robot / G. Orosco-Guerro, M. Velasco-Villa, E. Aranda-Bricaire // Proc., XI Latin-American Congress of Automatic Control, La Habana, Cuba, 2004.
	
	%\bibitem{peiffer69}
	%{\it Peiffer K.} Liapounov's second method applied to partial stability / K. Peiffer, N. Rouche // J. Mechanique.- 1969.- Vol. 8.- № 2.- Pp. 323-334. (Русский перевод: Сборник переводов. Механика. 1970. Т.6:124. С. 20-29).
	
	\bibitem{purwin06}
	{\it Purwin O.} Trajectory generation and control for four wheeled omnidirectional vehicles / O. Purwin, R. D'Andrea // Robotics and Autonomous Systems, 2006. V. 54(1).- Pp. 13-22.
	
	\bibitem{rondoni93}
	{\it Rondoni L.} Autocatalytic reactions as dynamical systems on the interval / L. Rondoni // J. Math. Phys.- 1993.- Vol. 34.- no. 11.- Pp. 5238-5251.
	
	\bibitem{sedaghat97}
	{\it Sedaghat H.} A class of nonlinear second order difference equations from macroeconomics / H. Sedaghat // Nonlinear Anal. Theory, Methods, Appl.- 1997.- Vol. 29.- no. 5.- Pp. 593-603.
	
	\bibitem{tchuente93}
	{\it Tchuente M.} Suites generees par une equation neuronale a memoire (Sequences generated by a neuronal recurrence equation with memory) / M. Tchuente, G. Tindo // C. R. Acad. Sci. Paris.- 1993.- Ser I.- vol. 317.- no. 6.- Pp. 625-630.
	
	\bibitem{sell67}
	{\it Sell G. R.} Nonautonomous differential equations and topological dynamics 1, 2 /
	G. R. Sell // Trans. Amer. Math. Soc.— 1967.— Vol. 127.— Pp. 241–283.
	
	\bibitem{simonovits85}
	{\it Simonovits A.} Chaotic dynamics of economic systems /
	A. Simonovits // Szigma.— 1985.— Vol. 18.— Pp. 267–277.
	
	\bibitem{vukobratovic79}
	{\it Vukobratovic M.} Vukobratovic M., Potkonjak V. Contribution to automatic forming of active chain models via Lagrangian form. J of Appl. Mech., N 1, 1979.
	
	\bibitem{wang12}
	{\it Wang J.} WANG,J.,HAMNER,S.,DELP,S., AND KOLTUN,V.2012.Optimizing locomotion controllers using biologically-based actuators and objectives. ACM Trans.on Graphics3 1,4,25.
	
	\bibitem{yang01}
	{\it Yang T.} Impulsive Control Theory.— Berlin, Heidelberg: Springer, 2001.
	
	\bibitem{yarza11}
	{\it Yarza A.} Yarza A., Santibanez V., Moreno-Valenzuela J. Uniform Global Asymptotic Stability of an Adaptive Output Feedback Tracking Controller for Robot Manipulators // Preprints of the 18th IFAC World Congress Milano (Italy) August 28 - September 2, 2011. P. 14590-14595. 
	
	
\end{thebibliography} 