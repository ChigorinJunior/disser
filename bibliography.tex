\begin{thebibliography}{10} \label{bibl}
	
	\bibitem{avetisian87}
	{\it Аветисян В. В.} Оптимизация режимов управления манипуляционными роботами с учетом энергозатрат / В.В. Аветисян, Л.Д. Акуленко, Н.Н. Болотник // Изв. АН СССР. Техническая кибернетика. 1987. №3.
	
	\bibitem{avetisian85}
	{\it Аветисян В. В.} Оптимальные программные движения двухзвенного манипулятора / В.В. Аветисян, Н.Н. Болотник, Ф.Л. Черноусько // Изв. АН СССР. 1985. Техническая кибернетика. № 3. С. 123-131.
	
	\bibitem{akulenko00}
	{\it Акуленко Л. Д.} Активное гашение колебаний крупногабаритных несущих конструкций посредством перемещения внутренних масс / Л.Д. Акуленко, Н.Н. Болотник, С.А. Кумакшев, А.А. Чернов // Известия РАН. Теор. и сист. упр. -- 2000. -- № 1. -- С. 135-145.
	
	\bibitem{akulenko88}
	{\it Акуленко Л. Д.} Анализ управлений упругого манипулятора с электромеханическими приводами / Л.Д. Акуленко, С.А. Михайлов // МТТ. 1008. № 1. С. 75-81.
	
	\bibitem{aleksandrov00}
	{\it Александров, В. В.} Оптимизация динамики управляемых систем / В. В. Александров, В. Г. Болтянский,
	С. С. Лемак и др.— М.: МГУ, 2000.— 303 с.
	
	\bibitem{anan89}
	{\it Ананьевский И.М.} О стабилизации некоторых регулируемых систем с последействием / И.М. Ананьевский, В.Б. Колмановский // Автоматика и телемеханика. – 1989. - № 9. – С. 34 – 42.
	
	\bibitem{anan892}
	{\it Ананьевский И.М.} Об управлении некоторыми механическими системами при неполной информации / И.М. Ананьевский, В.Б. Колмановский // Изв. АН СССР. Техническая кибернетика. 1989. № 1. С. 30-36.

	\bibitem{anan97}
	{\it Ананьевский И.М.} Управление механической системой с неизвестными параметрами посредством ограниченной силы / И.М. Ананьевский // Прикладная математика и механика, 1997. Т. 61, Вып. 1.
	
	\bibitem{anan972}
	{\it Ананьевский И.М.} Управление линейной механической системой с упругими элементами в условиях неопределенности / И.М. Ананьевский // Известия РАН. ТСУ. 1997. № 4.

	\bibitem{anan98}
	{\it Ананьевский И.М.} Управление двухмассовой системой с неизвестными параметрами / И.М. Ананьевский // Известия РАН. ТСУ. 1998. № 2. С. 72-82.

	\bibitem{ananev012}
	{\it Ананьевский, И. М.} Два подхода к управлению механической системой с неизвестными параметрами /
	И. М. Ананьевский // Изв. РАН. Теор. и сист. упр.— 2001.— № 2.— С. 39–47.
	
	\bibitem{ananev014}
	{\it Ананьевский, И. М.} Управление реономными механическими системами с неизвестными параметрами /
	И. М. Ананьевский // Докл. РАН.— 2001.— Т. 337, № 4.— С. 459–463.
	
	\bibitem{ananev95}
	{\it Ананьевский, И. М.} Метод декомпозиции в задаче управления динамической системой / И.М. Ананьевский, И.С. Добрынина, Ф.Л. Черноусько // Известия РАН. Теор. и сист. упр. -- 1995. - № 2. - С. 3-14.
	
	\bibitem{anan02}
	{\it Ананьевский, И. М.} Метод декомпозиции в задаче об отслеживании траектории механических систем / И.М. Ананьевский, С.А. Решмин // Известия 	РАН. Теория и системы управления. – 2002. - № 5. – С. 25 – 32.
	
	\bibitem{andr79} 
	{\it Андреев, А. С.} Об асимптотической устойчивости и неустойчивости неавтономных систем / А.С. Андреев // Прикладная математика и механика. – 	1979. – Т. 49, Вып. 5. – С. 		796 – 805.
	
	\bibitem{andr84}
	{\it Андреев, А. С.} Об асимптотической устойчивости и неустойчивости нулевого решения неавтономной системы / А.С. Андреев // Прикладная 		математика и механика. – 1984. – Т. 	48, Вып. 2. – С. 225 – 232.	
	
	\bibitem{andr87}
	{\it Андреев, А. С.} Об исследовании частичной асимптотической устойчивости и неустойчивости на основе предельных уравнений / А.С. Андреев // 		Прикладная математика и механика. – 1987. – Т. 51, Вып. 2. – С. 253-259.
	
	\bibitem{andr96} 
	{\it Андреев, А. С.} Об устойчивости положения равновесия неавтономной механической
	системы // Прикладная математика и механика. 1996. Т. 60, Вып. 3. С.388-396.

	\bibitem{andr05}
	{\it Андреев, А. С.} Устойчивость неавтономных функционально-дифференциальных уравнений / А.С. Андреев. – Ульяновск: УлГУ, 2005. – 328 с.

	\bibitem{andr09}
	{\it Андреев, А. С.} Метод функционалов Ляпунова в задаче об устойчивости функционально-дифференциальных уравнений / А.С. Андреев // Автоматика 	и телемеханика. – 2009, № 9. – С. 4 – 55.

	\bibitem{andr124}
	{\it Андреев, А. С.} Метод векторной функции Ляпунова в задаче об управлении систем 	
	с мгновенной обратной связью / А. С. Андреев, А. О. Артемова // Ученые записки
	Ульяновского государственного университета.— 2012.— № 1(4).— С. 15–19.
	
	\bibitem{andr126}
	{\it Андреев, А. С.} Об управлении движением голономной механической системы / А. С. Андреев,
	А. О. Артемова // Научно-технический вестник Поволжья.— 2012.— № 6.— С. 80–87.

	\bibitem{andr13_2}
	{\it Андреев, А. С.} Уравнения Вольтерра в моделировании ПИ- и ПИД-регуляторов / Андреев А.С., Благодатнов В.В., Кильметова А.Р. // Научно-		технический вестник Поволжья. – 2013. – № 1. – С.84-90
	
	\bibitem{andr0232}
	{\it Андреев, А. С.} Знакопостоянные функции Ляпунова в задачах об устойчивости /
	А. С. Андреев, Т. А. Бойкова // Механика твердого тела.— 2002.— № 32.— С. 109–116.

	\bibitem{andr044}
	{\it Андреев, А. С.} Об устойчивости неустановившегося решения механической системы /
	А. С. Андреев, Т. А. Бойкова // Прикладная математика и механика.— 2004.— Т. 68.— № 4.— С. 678–686.

	\bibitem{andr111}
	{\it Андреев, А. С.} Об устойчивости нулевого решения системы с разрывной правой частью /
	А. С. Андреев, О. Г. Дмитриева, Ю. В. Петровичева // Научно-технический вестник Поволжья.— 2011.— № 1.—
	С. 15–21.

	\bibitem{andr071}
	{\it Андреев, А. С.} Об устойчивости обобщенного стационарного движения механической системы в зависимости от действующих сил /
	А. С. Андреев, Р. Б. Зайнетдинов // Труды IX Международной Четаевской Конференции "Аналитическая механика, устойчивость и управление движением", 	посвященной 105-летию Н. Г. Четаева.— Иркутск: Сибирское отделение РАН.— 2007.— Т. 1.— С. 5–14.

	\bibitem{andr09_2}
	{\it Андреев, А. С.}  О стабилизации движений механических систем с
	переменными массами / Андреев А.С., Зайнетдинов Р.Б. //  Прикладная математика и механика.- 2009.- Т.73.- Вып.1.- С. 3-12.

	\bibitem{kudash152}
	{\it Андреев А. С.} О моделировании структуры управления для колес-ного робота с омни-колесами / Андреев А. С., Е. А. Кудашова // Автоматизация 	процессов управления №2 (40) 2015. Ульяновск: Автоматизация процессов управления, 2015 — с.114-121

	\bibitem{kudash14ek}
	{\it Андреев А. С.} О стабилизации движений механических систем управлениями различного типа / А.С. Андреев, Е.А. Кудашова, О.А. Перегудова // 		Тезисы докладов Международной конференции, посвященной 90-летию со дня рождения академика Н.Н. Красовского «Динамика систем и про-цессы 			управления», 15-20 сентября 2014г., Екатеринбург. – с. 35-37

	\bibitem{andr16_0}
	{\it Андреев А.С.} О стабилизации стационарных программных движений управляемой механической системы / А. С. Андреев, Д. С.  Макаров // Ученые записки УлГУ. Сер. "Математика и информационные технологии". -- Вып. 1(8). -- Ульяновск: УлГУ, 2016. -- С. 11-16.

	\bibitem{andr16}
	{\it Андреев А. С.}Об управлении манипуляторами без измерения скоростей / А.С. Андреев, Д.С. Макаров, О.А. Перегудова // Научно-технический 		вестник Поволжья. –- 2016. – № 6. – С.105-108. 

	\bibitem{andr13}
	{\it Андреев, А. С.} Об управлении двузвенным манипулятором с приводом / Андреев А.С., Макаров Д.С., Таджиев Д.А. // Научно-технический вестник 	Поволжья. – 2013. – № 5. – С.102-105.

	\bibitem{andr99}
	{\it Андреев, А. С.} Об устойчивости по части переменных неавтономного функционально-дифференциального уравнения / А.С. Андреев, С.В. 			Павликов // Прикладная математика и механика. – 1999. – Т. 63, Вып. 1. – С. 3 – 12.
	
	\bibitem{andr055}
	{\it Андреев, А. С.} К методу сравнения в задачах об асимптотической устойчивости /
	А. С. Андреев, О. А. Перегудова // Доклады Академии наук.— 2005.— Т. 400, № 5.—
	С. 621–624.
	
	\bibitem{andr06}
	{\it Андреев, А. С.} К методу сравнения в задачах об асимптотической устойчивости / А.С. Андреев, О.А. Перегудова // Прикладная математика и 		механика. – 2006. – Т. 70, Вып. 	6. – С. 965 – 976.

	\bibitem{andr14vspu}
	{\it Андреев, А. С.} О стабилизации программных движений голономной механической системы /
	А. С. Андреев, О. А. Перегудова // XII Всероссийское совещание по проблемам управления ВСПУ-2014. Институт проблем управления им. В. А. 			Трапезникова РАН.— 2014.— С. 1840–1843.

	\bibitem{andr141}
	{\it Андреев, А. С.} Вектор-функции Ляпунова в задачах о стабилизации движениц управляемых систем /
	А. С. Андреев, О. А. Перегудова // Журнал Средневолжского математического общества.— 2014.— Т. 16, № 1.—
	С. 32–44.

	\bibitem{andr15} 
	{\it Андреев, А. С.} Об управлении двухзвенным манипулятором с упругими шарнирами / Андреев А.С., Перегудова О.А. // Нелинейная динамика. – 		2015. – Т. 11, № 2. – С.267-277

	\bibitem{andr15}
	{\it Андреев, А. С.} Об управлении движением колесного мобильного робота /
	А. С. Андреев, О. А. Перегудова // Прикладная математика и механика.— 2015.— Т. 79.— № 4.— С. 451–462.

	\bibitem{andr02}
	{\it Андреев, А. С.} Об устойчивости обобщенного стационарного движения / Андреев А.С., Ризито К. // Прикладная математика и механика. 2002. Т.		66. Вып.3. С.339-349 

	\bibitem{andr078}
	{\it Андреев, А. С.} О стабилизации движения нестационарной управляемой системы /
	А. С. Андреев, В. В. Румянцев // Автоматика и телемеханика.— 2007.— № 8.—
	С. 18–31.
	
	\bibitem{andr98}
	{\it Андреев, А. С.} Предельные уравнения в задаче об устойчивости функционально-дифференциального уравнения / Андреев А.С., Хусанов Д.Х. // 		Дифференциальные уравнения. – 1998. – Т. 34, № 4. – С. 435 – 440.
	
	\bibitem{andr98_2} 
	{\it Андреев, А. С.} К методу функционалов Ляпунова в задаче об
	асимптотической устойчивости и неустойчивости / Андреев А.С., Хусанов Д.Х. // Дифференциальные уравнения. – 1998. – Т. 34, №7. С.876-885.
	
	\bibitem{artyomova126}
	{\it Артемова, А. О.} Моделирование управляемого движения двузвенного манипулятора на подвижном основании /
	А. О. Артемова // Научно-технический вестник Поволжья.— 2012.— № 6.
	
	\bibitem{artyomova135}
	{\it Артемова, А. О.} Об управлении пространственным движением многозвенного манипулятора на подвижном основании /
	А. О. Артемова, Е. Э. Звягинцева, Е. А. Кудашова // Научно-технический вестник Поволжья.— 2013.— № 5.— С. 106–109.
	
	\bibitem{afan03}
	{\it Афанасьев, В. Н.} Математическая теория конструирования систем управления: Учеб. для вузов / В.Н. Афанасьев, В.Б. Колмановский, В.Р. Носов 	– 3-е изд., испр. и доп. М.: Высш. шк., 2003. – 614 с.

	\bibitem{balak86}
	{\it Балакирева Т.Н.} Управление движением сборочного робота с переменной динамической моделью и ограничениями на нормальные силы // Балакирева 	Т.Н., Воробьев Е.И. / Изв. АН СССР. Мех. тверд. тела 1986, № 2, 99-102 – РЖМех, 1986, 8А166.

	\bibitem{belousov02}
	{\it Белоусов И.Р.} Формирование уравнений динамики роботов-  манипуляторов / И.Р. Белоусов.- М.: ИПМ им. М.В.Келдыша РАН, 2002
	
	\bibitem{borcov86}
	{\it Борцов, Ю. А.} Автоматические системы с разрывным управлением / Ю. А. Борцов,
	И. Б. Юнгер.— Л.: Энергоатомиздат. Ленингр. отделение, 1986.— 168 с.
	
	\bibitem{bulg84}
	{\it Булгаков, Н. Г.} Знакопостоянные функции в теории устойчивости /
	Н. Г. Булгаков.— Минск: Университетское, 1984.— 78 с.
	
	\bibitem{burkov98}
	{\it Бурков И.В} Стабилизация натуральной механической системы без измерения ее скоростей с приложением к управлению твердым телом/ И. В. 		Бурков // Прикладная математика и механика. 1998. Т. 62, Вып. 6. С. 923-933.
	
	\bibitem{burkov88}
	{\it Бурков И.В} Динамика пространственного упругого манипулятора с распределенными параметрами / Бурков И.В., Заремба А.П. // Изв. АН СССР. 		Мех. тверд. тела, 1988, № 3, 43 – 52 – РЖМех, 1988, 11А199.
	
	\bibitem{burkov97} 
	{\it Бурков И.В} Стабилизация положения лагранжевой системы с упругими элементами при ограничениях на управление с измерением и без 			измерения скоростей / И.В. Бурков, Л.Б. Фрейдович // Прикладная математика и механика. – 1997. – Т. 61, Вып. 3. – С. 447 – 456.
	
	\bibitem{burkov96} 
	{\it Бурков И.В} Стабилизация положения упругого робота ПД-регулятором / И.В. Бурков, А.А. Первозванский, Л.Б. Фрейдович // ТСУ. 1996. № 6. С. 159-165.
	
	\bibitem{vasil819}
	{\it Васильев, С. Н.} Метод сравнения в анализе систем 1, 2 / С. Н. Васильев // Дифференциальные уравнения.— 1981.— Т. 17, № 9.— С. 1562–1573.
	
	\bibitem{verbuk89}
	{\it Вербюк В.Е.} Динамика и оптимизация роботехинических систем / Вербюк В.Е. // Киев: Наук. думка, 1989, 187 с. – РЖМех, 1989, 7А193.
	
	\bibitem{veresh75}
	{\it Верещагин А.Ф.} Принцип наименьшего принуждения Гаусса для моделирования на ЭВМ динамики роботов-манипуляторов / Верещагин А.Ф. // Докл. АН 	СССР, 1975, 220, № 1, 51 – 53.
	
	\bibitem{girko}
	{\it Гирко В.Л.} Адаптивный подход к управлению движением манипулятора / В.Л. Гирко, Ю.В. Крак // Доклады АН УССР, А. №12, 1985.
	
	\bibitem{golubev05}
	{\it Голубев А. Е.} Стабилизация нелинейных динамических систем с использованием оценки состояния системы асимптотически наблюдателем / А. Е. 		Голубев, А. П. Крищенко, С. Б. Ткачев // Автоматика и телемеханика.— 2005.— № 7.— С. 3–42.
	
	\bibitem{gusev}
	{\it Гусев С.В.} Алгоритм адаптивного управления роботом-манипулятором / С.В. Гусев, В.А. Якубович // Автоматика и телемеханика.— 1980.— № 9.
	
	\bibitem{dobr95}
	{\it Добрынина И. С.} Моделирование динамики манипуляционных роботов с применением метода декомпозиции управления / И. С. Добрынина // Известия РАН. Техн. и кибернет.— 1995.— № 4.— С. 246–256.

	\bibitem{dunsk83}
	{\it Дунская Н. В.} Адаптивное управление манипулятором. Алгоритмы обучения движению / Н.В. Дунская, Е.С. Пятницкий // Автоматика и телемеханика. 1983. № 2. 
	
	\bibitem{dunsk88}
	{\it Дунская Н. В.} Метод потенциала цели в задачах синтеза управления манипуляционными роботами / Н.В. Дунская, Е.С. Пятницкий // Известия АН СССР. Техн. кибернетика. 1988. № 4. С. 14-24.
	
	\bibitem{dunsk94}
	{\it Дунская Н. В.} Метод синтеза управления упругими системами / Н.В. Дунская, Е.С. Пятницкий // ДАН. 1994. Т. 338. № 2.
	%\bibitem{dobr94}
	%{\it Добрынина И. С.} Компьютерное моделирование управлением движения системы связных тел / И. С. Добрынина, И. И. Карпов, Ф. Л. Черноусько  // 	Изв. РАН. Техн. и кибернет.— 1994.— № 1.— С. 167–180.
	%\bibitem{dorf04}
	%{\it Дорф Р.} Современные системы управления /
	%Р. Дорф, Р. Бишоп; Пер. с англ. Б. И. Копылова.— М.: Лаборатория Базовых знаний, 2004.— 832 с.
	%\bibitem{druzh074}
	%{\it Дружинин, Э. И.} Об устойчивости прямых алгоритмов расчета программных управлений
	%в нелинейных системах / Э. И. Дружинин // Известия РАН. Теория и системы
	%управления.— 2007.— Т. 3, № 4.— С. 14–20.
	%\bibitem{dixta03}
	%{\it Дыхта, В. А.} Оптимальное импульсное управление с приложениями / В. А. Дыхта, О. Н. Самсонюк // М.:
	%Физматлит, 2003.
	%\bibitem{emel06}
	%{\it Емельянов, С. В.} Избранные труды по теории управления / С. В. Емельянов.— М.:
	%Наука, 2006.— 450 с.
	%\bibitem{zavriev90}
	%{\it Завриев С. К.} Прямой метод Ляпунова в исследовании притяжения траекторий конечно-разностных включений / С. К. Завриев,  А. Г. 			Перевозчиков // Журнал вычислительной математики и математической физики.— 1990.— Т. 30, № 1.— С. 22–32.
	
	\bibitem{zak}
	{\it Зак В.Л.} Моделирования динамики манипуляторов с упругими шарнирами / В.Л. Зак, Г.У. Перумов, Н.Н. Рогов // Известия АН СССР, Механика твердого тела, №3, 1987.
	
	\bibitem{kudash131}
	{\it Звягинцева Е. Э.} Об управлении механической системой с циклическими координатами / Е. Э. Звягинцева, Е. А. Кудашова //
	Научно-технический вестник Поволжья №1 2013. Казань: Научно-технический вестник Поволжья, 2013 - с. 217-221
	
	\bibitem{ignat72}
	{\it Игнатьев М.Б.} Алгоритмы управления роботами-манипуляторами. / Игнатьев М.Б., Кулаков Ф.М., Покровский А.М. //  Л.: Машиностроение, 1972. – 	360 с.
	
	\bibitem{karap98}
	{\it Карапетян, А. В.} Устойчивость стационарных движений / А. В. Карапетян.—
	М.: УРСС, 1998.
	
	\bibitem{karap98_2}
	{\it Карапетян А.В.} О влиянии диссипативных и постоянных сил на вид и устойчивость стационарных движений механических систем с циклическими 		координатами / А.В. Карапетян, И.С. Лагутина // Прикладная математика и механика. – 1998. – Т.62, Вып. 4. – С. 539 – 547.

	\bibitem{karap83}
	{\it Карапетян А.В.} Устойчивость консервативных и диссипативных систем / А.В. Карапетян, В.В. Румянцев – М.: ВИНИТИ, 1983. – (Итоги науки  		техники. Сер. Общая механика; Т.6).
	
	\bibitem{kirichenko87}
	{\it Кириченко Н. Ф.} Математическое моделирование механики и процессов управления манипуляционными роботами / Н.Ф. Кириченко, Ю.В. Крак, Р.А. Сорока // Известия АН СССР. Техническая кибернетика. 1987. №3.
	
	\bibitem{kobr85}
	{\it Кобринский А.А.} Манипуляционные системы роботов / Кобринский А.А., Кобринский А.Е. // М.: Наука, 1985, 344 с.
	
	\bibitem{kozlov}
	{\it Козлов В. В.} Динамика управления роботами / В.В. Козлов, В.П. Макарычев, А.В. Тимофеев, Е.И. Юревич // М.: Наука, 1984
	
	\bibitem{kolmanov965}
	{\it Колмановский B. Б.} Устойчивость дискретных уравнений Вольтерра / В. Б. Колмановский // Доклады Академии наук.— 1996.— Т. 349, № 5.— С. 610–614.
	
	\bibitem{kra591}
	{\it Красовский, Н. Н.} Некоторые задачи теории устойчивости движения. / Красовский Н.Н.// М.:
	Физматгиз, 1959. 211 с.
	
	\bibitem{krasovsk63}
	{\it Красовский, Н. Н.} О стабилизации неустойчивых движений дополнительными силами при неполной обратной связи / Н. Н. Красовский,
	// Прикладная математика и механика.— 1963.— Т. XXVII, № 4.— С. 641–663.
	
	\bibitem{krasovsk664}
	{\it Красовский, Н. Н.} Проблемы стабилизации управляемых движений / Н. Н. Красовский
	// Малкин, И. Г. Теория устойчивости движения. Доп. 4 / И. Г. Малкин.— М.:
	Наука, 1966.— С. 475–514.
	
	\bibitem{krutko87_2}
	{\it Крутько, П. Д.} Обратные задачи динамики управляемых систем: линейные модели / Крутько П.Д. // М.: Наука, 1987, 304 с.
	
	\bibitem{krutko88}
	{\it Крутько, П. Д.} Обратные задачи динамики управляемых систем: нелинейные модели / Крутько П.Д. // М.: Наука, 1988, 328 с.
	
	\bibitem{krutko951}
	{\it Крутько, П. Д.} Управление движением Эйлеровых систем. Синтез алгоритмов методом обратных задач динамики / П.Д. Крутько // ТСУ. 1995. № 1 С. 34-53.
	
	\bibitem{krutko952}
	{\it Крутько, П. Д.} Управление движением лагранжевых систем. Синтез алгоритмов методом обратных задач динамики / П.Д. Крутько // ТСУ. 1995. № 6 С. 19-37.
	
	\bibitem{krutko873}
	{\it Крутько, П. Д.} Метод обратных задач динамики в теории конструирования алгоритмов
	управления манипуляционных роботов. Задача стабилизации / П. Д. Крутько, Н. А. Лакота
	// Изв. АН СССР. Техническая кибернетика.— 1987.— № 3.— С. 23–30.

	\bibitem{krutko79}
	{\it Крутько П.Д.} Кинематические алгоритмы управления движением манипуляционных роботов / Крутько П.Д., Попов Е.П. // Известия АН СССР. Техническая кибернетика. 1979. № 4. С. 77-86.

	\bibitem{krutko80}
	{\it Крутько П.Д.} Построение алгоритмов управления движением манипуляционных роботов / Крутько П.Д., Попов Е.П. // Докл. АН СССР, 1980, 255, 		№ 1, 40 – 43.
	
	\bibitem{kulak89}
	{\it Кулаков Ф.М.} Супервизорное управление манипуляционными роботами / Кулаков Ф.М. // М.: Наука, 1989, 448 с.
	
	\bibitem{kuleshov71}
	{\it Кулешов, В. С.} Динамика систем управления манипуляторами / В. С. Кулешов,
	Н. А. Лакота.— М.: Энергия, 1971.— 304 с.
	
	\bibitem{lyapunov07}	
	{\it Ляпунов А. М.} Избранные труды. Работы по теории устойчивости / А.М. Ляпунов – М.: Наука, 2007. – 574 с.

	\bibitem{makarov15}
	{\it Макаров Д.С.} О стабилизации программных движений трехзвенного манипулятора / Д.С. Макаров, Л.С. Тахтенкова // Тезисы международной 		конференции по математической теории управления и механике, 3-7 июля 2015г., г. Суздаль. – с. 
	
	\bibitem{makarov16}
	{\it Макаров Д.С.} Программа моделирования динамики трехзвенного манипулятора с заданием закона управления / Д. С. Макаров // Ученые записки УлГУ. Сер. "Математика и информационные технологии". -- Вып. 1(8). -- Ульяновск: УлГУ, 2016. -- С. 73-75.
	
	\bibitem{malikov98}
	{\it Маликов, А. И.} Вектор-функции Ляпунова в анализе свойств систем со структурными изменениями / А. И. Маликов, В. М. Матросов // 			Дифференциальные уравнения.— 1998.— № 2.— С. 47–54, 530 с.
	
	\bibitem{malkin66}
	{\it Малкин, И. Г.} Теория устойчивости движения / И. Г. Малкин.— М.: Наука, 1966.—
	530 с.
	
	\bibitem{markeev99}
	{\it Маркеев, А. П.} Теоретическая механика / А. П. Маркеев.— М.: ЧеРо, 1999.— 569 с.
	
	\bibitem{matrosov01}
	{\it Матросов, В. М.} Метод векторных функций Ляпунова: анализ динамических свойств нелинейных систем / В. М. Матросов.— М.: Физматлит, 2001.— 		380 с.

	\bibitem{matuhun893}
	{\it Матюхин, В. И.} Устойчивость движений манипуляционных роботов в режиме декомпозиции
	/ В. И. Матюхин // Автоматика и телемеханика.— 1989.— № 3.— С. 33–44.
	
	\bibitem{matuhun93}
	{\it Матюхин, В. И.} Устойчивость движений манипулятора при учете постоянно действующих возмущений / В. И. Матюхин // Автоматика и телемеханика. -- 1993. -- № 11. С. 124-134.
	
	\bibitem{matuhun961}
	{\it Матюхин, В. И.} Сильная устойчивость движений механических систем
	/ В. И. Матюхин // Автоматика и телемеханика.— 1996.— № 1.— С. 37–56.
	
	\bibitem{matuhun96}
	{\it Матюхин, В. И.} Устойчивость движений манипулятора при учете слабой динамики управляющих устройств / В. И. Матюхин // Автоматика и телемеханика.— 1996.— № 4.— С. 24–38.
	
	\bibitem{matuhun97}
	{\it Матюхин, В. И.} Непрерывные универсальные законы управления манипуляционным роботом
	/ В. И. Матюхин // Автоматика и телемеханика - 1997. - № 4. - С. 69-82.
	
	\bibitem{matuhun972}
	{\it Матюхин, В. И.} Стабилизация движений лагранжевых систем за конечное время переходного процесса / В. И. Матюхин // Докл. РАН. -- 1997. -- Т.353. № 4. С. 484-487.
	
	\bibitem{matuhun973}
	{\it Матюхин, В. И.} Стабилизация движений упругого манипулятора / В. И. Матюхин // Автоматика и телемеханика. -- 1997. - № 9. - С. 15-30.
	
	\bibitem{matuhun974}
	{\it Матюхин, В. И.} Эффект движения звена упругого манипулятора как абсолютно твердого тела / В. И. Матюхин // МТТ. -- 1997. - № 6. - С. 49-59.
	
	\bibitem{matuhun98}
	{\it Матюхин, В. И.} Устойчивость многообразий управляемых движений манипулятора
	/ В. И. Матюхин // Автоматика и телемеханика - 1998. - № 4. - С. 47-56.
	
	\bibitem{matuhun982}
	{\it Матюхин, В. И.} Метод медленных переменных в задаче управления движением упругого манипулятора
	/ В. И. Матюхин // Прикладная математика и механика - 1998. - Т. 62, вып. 3. - С. 61-71.
	
	\bibitem{matuhun99}
	{\it Матюхин, В. И.} Об условиях допустимости описания движения звена манипулятора как абсолютно твердого тела
	/ В. И. Матюхин // Известия РАН. ТСУ. С. 124-132. 1999.
	
	\bibitem{matuhun01}
	{\it Матюхин, В. И.} Универсальные законы управления механическими системами /
	В. И. Матюхин.— М.: МАКС Пресс, 2001.— 252 с.

	\bibitem{matuhun03}
	{\it Матюхин, В. И.} Интегрирование на ЭВМ движений разрывной механической системы /
	В. И. Матюхин. // Автоматика и телемеханика. -- 2003. -- № 9. С. 42-59.

	\bibitem{matuhun05}
	{\it Матюхин, В. И.} Управляемость механических систем при учете динамики приводов / В. И. Матюхин // Автоматика и телемеханика. 2005. № 12. С. 75-92.
	
	\bibitem{matuhun06}
	{\it Матюхин, В. И.} Управление колесной системой при учете погрешностей измерения состояния / В. И. Матюхин // Автоматика и телемеханика. 2006. № 11. С. 41-60.
	
	\bibitem{matuhun09}
	{\it Матюхин, В. И.} Управление механическими системами / В. И. Матюхин.— М.: Физматлит,
	2009.— 320 с.
	
	\bibitem{matuhun10}
	{\it Матюхин, В. И.} Управление движением манипулятора / В. И. Матюхин.— М.: ИПУ
	РАН, 2010.— 96 с.
	
	\bibitem{matuhun899}
	{\it Матюхин, В. И.} Управление движением манипуляционных роботов на принципе декомпозиции при учете динамики приводов
	/ В. И. Матюхин, Е. С. Пятницкий // Автоматика и телемеханика.— 1989.— № 9.— С. 67–82.
	
	\bibitem{matuhun892}
	{\it Матюхин, В. И.} Синтез систем управления многозвенными механизмами на принципе декомпозиции при учете динамики приводов / В. И. Матюхин, Е. С. Пятницкий // Машиноведение. — 1989.— № 3.— С. 41–48.
	
	\bibitem{matuhun048}
	{\it Матюхин, В. И.} Управляемость механических систем в классе управлений, ограниченных вместе с производной
	/ В. И. Матюхин, Е. С. Пятницкий // Автоматика и телемеханика.— 2004.— № 8.— С. 14–38.
	
	\bibitem{medvedev78}
	{\it Медведев, В. С.} Системы управления манипуляционных роботов / В. С. Медведев,
	А. Г. Лесков, А. С. Ющенко.— М.: Наука, 1978.— 416 с.
	
	%\bibitem{mishkis67}
	%{\it Мышкис, А. Д.} Система с толчками в заданные моменты времени / А. Д. Мышкис
	%// Математ. сборник.— 1967.— Т. 74 (116), № 2.— С. 202–208.
	
	\bibitem{ohocim}
	{\it Охоцимский Д.Е.} Механика и управление движением автоматического шагающего аппарата / Д.Е. Охоцимский, Ю.Ф. Голубев // М.: Наука, 1984.
	
	\bibitem{pavl06}
	{\it Павликов С.В.} Метод функционалов Ляпунова в задах устойчивости / С.В. Павликов – Набережные Челны: Ин-т управления, 2006. 253 с.
	
	\bibitem{pavl07}
	{\it Павликов С.В.} К задаче о стабилизации управляемых механических систем / С.В. Павликов // Автоматика и телемеханика. – 2007. - № 9. – С. 16 	– 26.
	
	\bibitem{pavl07_2}
	{\it Павликов С.В.} О стабилизации движений управляемых систем с запаздывающим регулятором / С.В. Павликов // Доклады Академии наук. – 2007. – 		Т. 412, № 2. – С. 176 – 178.
	
	\bibitem{pavlovsk}
	{\it Павловский Ю.Н.} Теория факторизации и декомпозиция управляемых динамических систем и ее приложения / Ю.Н. Павловский // Известия АН СССР. Техническая кибернетика - 1984. № 2. С. 45-57.
	
	\bibitem{pereg06}
	{\it Перегудова, О. А.} Функции Ляпунова вида векторных норм в задачах устойчивости / О.А. Перегудова // Уч. зап. Ульян. гос. ун-та. Сер. 	«Фундаментальные проблемы математики и механики». – Ульяновск: Изд-во УлГУ, 2006. – Вып. 1(16). – С. 43 – 51
	
	\bibitem{pereg07}
	{\it Перегудова, О. А.} Знакопостоянные функции Ляпунова в задаче об устойчивости функционально-дифференциальных уравнений / О.А. Перегудова // 	Международный сборник «Проблемы нелинейного анализа в инженерных системах». – 2007. – Т. 13, №2(28). – С. 97 – 108.

	\bibitem{peregudova07}
	{\it Перегудова, О. А.} Уравнения сравнения в задачах об устойчивости движения / О. А. Перегудова // Автоматика и телемеханика.—  2007.— № 9.— 		С. 56–63.
	
	\bibitem{pereg08}
	{\it Перегудова, О. А.} Логарифмические матричные нормы в задачах устойчивости движения / О.А. Перегудова // Прикладная математика и механика. – 	2008. – Т. 72, Вып. 3. – С. 410 – 420.
	
	\bibitem{pereg08_3}
	{\it Перегудова, О. А.} Развитие метода функций Ляпунова в задаче устойчивости функционально-дифференциальных уравнений / О.А. Перегудова // 		Дифференциальные уравнения. – 2008. – Т. 44, № 12. – С. 1638 – 1647.

	\bibitem{peregudova09}
	{\it Перегудова, О А.} Метод сравнения в задачах устойчивости и управления движениями
	механических систем / О. А. Перегудова.— Ульяновск: Изд-во УлГУ, 2009.— 253 с.

	\bibitem{pereg09_5}
	{\it Перегудова, О.А.} О стабилизации движений неавтономных механических систем // Прикладная математика и механика. – 2009. – Т.73. – Вып.2. – 	С.176-188.

	\bibitem{peregudova092}
	{\it Перегудова, О.А.} О стабилизации движений неавтономных механических систем / О. А. Перегудова // Прикладная математика и механика.—  2009.	— Т. 72.— Вып. 4.— С. 620.
	
	\bibitem{pereg09}
	{\it Перегудова О.А.} К задаче слежения для механических систем с запаздыванием в управлении / О.А. Перегудова // Автоматика и телемеханика. –  	2009. - № 5. – С. 95 – 105.
	
	\bibitem{pereg11}
	{\it Перегудова, О. А.} О стабилизации программного движения нелинейных механических систем при помощи кусочно-непрерывных управлений // 		Автоматизация процессов управления. 2011. №1(23) . – С. 78-82.
	
	\bibitem{peregudova14}
	{\it Перегудова, О. А.} Синтез управления двухзвенным манипулятором / Перегудова О.А., Макаров Д.С. // Автоматизация процессов управления. – 		2014. – № 4(38). – С. 36–41.
	
	\bibitem{pereg15_2}
	{\it Перегудова, О. А.} Синтез управления трехзвенным манипулятором / Перегудова О.А., Макаров Д.С. // Автоматизация процессов управления. – 		2015. – № 2. – С.109-113.

	\bibitem{peregudova16}
	{\it Перегудова, О. А.} Стабилизация программного движения манипуляционных роботов на основе измерения координат звеньев / О.А. Перегудова, Д.С. 	Макаров // СВМО. — 2016. — Том 18. — № 4. — С. 46-51.
	
	\bibitem{peregudova16_2}
	{\it Перегудова, О. А.} О стабилизации программных позиций голономной механической системы без измерения скоростей /  О. А. Перегудова, Д. С. Макаров // Ученые записки УлГУ. Сер. "Математика и информационные технологии". -- Вып. 1(8). -- Ульяновск: УлГУ, 2016. -- С. 76-85.

	\bibitem{peregudova13}
	{\it Перегудова, О. А.} О стабилизации нелинейных систем с кусочно-постоянным управлением при помощи метода бэкстеппинга / О. А. Перегудова, К. 	В. Пахомов // Автоматизация процессов управления.—  2013.— № 4(34).

	\bibitem{petrov79}
	{\it Петров, Б. Н.} Построение алгоритмов управления как обратная задача динамики / Б. Н. Петров, П. Д. Крутько, Е. П. Попов // Докл. АН СССР.— 	1979.— № 5.— С. 1078–1081.
	
	\bibitem{pogor05}
	{\it Погорелов Д.Ю.} Современные алгоритмы компьютерного синтеза уравнений движения систем тел / Д.Ю. Погорелов // Известия РАН. Теория и 		системы управления. – 2005. - № 4. – С. 5 – 15.
	
	\bibitem{pol76}
	{\it Пол Р.} Моделирование, планирвоание тракторий и управление движением робота-манипулятора / Пол. Р. // Перевод с англ. – М.: Наука, 1976. – 103 с.
	
	\bibitem{popov781}
	{\it Попов, Е. П.} Манипуляционные роботы. Динамика и алгоритмы. / Е. П. Попов, А. Ф. Верещагин,
	С. Л. Зенкевич.— М.: Наука, 1978.— 398 с.

	\bibitem{popov78}
	{\it Попов, Е. П.} Системы управления манипуляционных роботов / Е. П. Попов, А. Ф. Верещагин,
	С. Л. Зенкевич.— М.: Наука, 1978.— 400 с.
	
	\bibitem{popov76}
	{\it Попов Э.В.} Алгоритмические основы интеллектуальных роботов и искусственного интеллекта / Попов Э.В., Фридман Г.Р. // М.: Наука, 	1976. – 	456 с.
	
	\bibitem{pjatnic873}
	{\it Пятницкий, Е. С.} Синтез систем управления манипуляционными роботами на принципе декомпозиции
	/ Е. С. Пятницкий // Известия АН СССР. Техническая кибернетика.—
	1987.— № 3.— С. 92–99.
	
	\bibitem{pjatnic882}
	{\it Пятницкий, Е. С.} Принцип декомпозиции в управлении механическими системами
	/ Е. С. Пятницкий // ДАН СССР.— 1988.— Т.— 300.— № 2.— С. 300-303.
	
	\bibitem{pjatnic89}
	{\it Пятницкий, Е. С.} Синтез иерархических систем управления механическими объектами на принципе декомпозиции
	/ Е. С. Пятницкий // Автоматика и телемеханика. 1989. № 1. С. 87-99; № 2. С. 71-86.
	
	\bibitem{pjatnic937}
	{\it Пятницкий, Е. С.} Синтез систем стабилизации программных движений нелинейных объектов управления
	/ Е. С. Пятницкий // Автоматика и телемеханика.— 1993.— № 7.— С. 19-37.
	
	\bibitem{pjatnic96}
	{\it Пятницкий, Е. С.} Критерий полной управляемости классов механических систем с ограниченными управлениями 
	/ Е. С. Пятницкий // Прикладная математика и механика. — 1996. — Т. 60, вып. 5. — С. 707-718.
	
	\bibitem{remshin97}
	{\it Ремшин, С. А.} Синтез управления двузвенным манипулятором /
	С. А. Ремшин // Известия РАН. Теор. и сист. упр.— 1997.— № 2.— С. 146–150.
	
	\bibitem{remshin98}
	{\it Ремшин, С. А.} Синтез управления в нелинейной динамической систему на основе декомпозиции /
	С. А. Ремшин, Ф. Л. Черноусько // Прикладная математика и механика. — 1998.— Т. 62, вып. 1.— С. 121–128.
	
	\bibitem{rumyncev07}
	{\it Румянцев, В. В.} О стабилизации движения нестационарной управляемой системы / В.В. Румянцев, А.С. Андреев // Доклады Академии наук. – 2007. 	– Т. 416, № 5. – С. 627-629.

	\bibitem{rumyncev87}
	{\it Румянцев, В. В.} Устойчивость и стабилизация движения по отношению к части переменных /
	В. В. Румянцев, А. С. Озиранер. — М.: Наука, 1987,- 253 с.
	
	\bibitem{rush80}
	{\it Руш Н.} Прямой метод Ляпунова в теории устойчивости / Н. Руш, П. Абетс, М. Лалуа – М.: Мир, 1980. – 300 с.
	
	\bibitem{tad12}
	{\it Таджиев Д.А.} Об управлении движением голономной системы с учетом динамики приводов / Д.А. Таждиев // Научно-технический вестник Поволжья. 	– 2012. - № 6. 
	
	\bibitem{timof86}
	{\it Тимофеев А.В.} Управление роботами / Тимофеев А.В. // Л.: Ленингр. ун-т, 1986, 276 с.
	
	\bibitem{timof89}
	{\it Тимофеев А.В.} Адаптивное управление роботами / Тимофеев А.В. // Известия АН СССР. Техническая кибернетика. 1989. № 1.
	
	\bibitem{timof96}
	{\it Тимофеев А. В.} Устойчивость и стабилизация программного движения робота-манипулятора /
	А. В. Тимофеев, Ю. В. Экало // Автоматика и телемеханика.— 1996.— № 10.
	
	\bibitem{utkin74}
	{\it Уткин В.И.} Скользящие режимы и их применения в системах с переменной структурой / В.И. Уткин // М.: Наука, 1974.
	
	\bibitem{utkin81}
	{\it Уткин В.И.} Скользящие режимы в задачах оптимизации управления / В.И. Уткин // М.: Наука, 1981. 368 с.
	
	\bibitem{halil09}
	{\it Халил Х. К.} Нелинейные системы / Х. К. Халил.— М.: Ижевск: НИЦ "Регулярная и хаотическая динамика", Институт клмпьютерных исследований, 		2009.— 832 с.
	
	\bibitem{heil84}
	{\it Хейл Дж.} Хейл Дж. Теория функционально-дифферепнциальных уравнений. М.: Мир,
	1984. 421 с.
	
	\bibitem{filip85}
	{\it Филиппов, А. Ф.} Дифференциальные уравнения с разрывной правой частью /
	А. Ф. Филиппов. — М.: Наука, 1985.
	
	\bibitem{finogenko07}
	{\it Финогенко, И. А.} О задачах слежения, управляемости и стабилизации для механических систем с использованием комбинаций разрывных обратных 		связей и импульсных управлений / И. А. Финогенко // Труды IX Международной Четаевской Конференции "Аналитическая механика, устойчивость и 		управление движением", посвященной 105-летию Н. Г. Четаева.— Иркутск: Сибирское отделение РАН.— 2007.— Т. 2.— С. 299–307.
	
	\bibitem{form74}
	{\it Формальский А.М.} Управляемость и устойчивость систем с ограниченными ресурсами / Формальский А.М. // М.: Наука, 1974. – 367 с.
	
	\bibitem{frolov88}
	{\it Фролов К.В.} Механика промышленных роботов. Кн. 1. Кинематика и динамика / Фролов К.В., Воробьев Е.И. // М.: Высш. шк., 1988. – 304 с.
	
	\bibitem{frolov88_2}
	{\it Фролов К.В.} Механика промышленных роботов. Кн. 2. Расчет и проектирование механизмов. / Фролов К.В., Воробьев Е.И. // М.: Высш. шк., 1988. 	– 367 с.
	
	\bibitem{frolov88_3}
	{\it Фролов К.В.} Механика промышленных роботов. Кн. 3. Основы конструирования / Фролов К.В., Воробьев Е.И. // М.: Высш. шк., 1988. – 383 с.
	
	\bibitem{fu_89}
	{\it Фу К.} Робототехника / Фу К., Гонсалес Р., Ли К. // Перевод с англ. – М.: Мир, 1989. – 624 с.
	
	\bibitem{chernou81}
	{\it Черноусько, Ф. Л.} Динамика управляемых движений упругого манипулятора / Черноусько Ф.Л. // Изв. АН СССР. Техн. Кибернет. 1981. №5. С. 		142-152.
	
	\bibitem{chernou90}
	{\it Черноусько, Ф. Л.} Декомпозиция и субоптимальное управление в динамических системах / Черноусько Ф.Л. // Прикладная математика и механика - 1990. - Т. 54, вып. 6. - С. 883-893.

	\bibitem{chernou06}
	{\it Черноусько, Ф. Л.} Методы управления нелинейными механическими системами /
	Ф. Л. Черноусько, И. М. Ананьевский, С. А. Решмин.— М.: Физматлит, 2006.— 326 с.
	
	\bibitem{chernou89}
	{\it Черноусько, Ф. Л.} Манипуляционные роботы: динамика, управление, оптимизация /
	Ф. Л. Черноусько, Н. Н. Болотник, В. Г. Градецкий.— М.: Физматлит, 1989.— 368 с.

	
	\bibitem{shahin98}
	{\it Шахинпур М.} Курс робототехники / Шахинпур М. // Перевод с англ. – М.: Мир, 1990. – 527 с. 
	
	\bibitem{jurevic05}
	{\it Юревич, Е. И.} Основы робототехники / Е. И. Юревич.— 2-е изд.— СПб.:
	БХВ-Петербург, 2005.— 416 с.

	\bibitem{luka88} 
	{\it Alessandro De Luca} Dynamic control of robots with joint elasticity / Alessandro De Luca // Proceedings of the IEEE
	International Conference on Robotics and Automation. 1988. P. 152-158.

	\bibitem{luka00} 
	{\it Alessandro De Luca} Feedforward/Feedback laws for the control of flexible robots / Alessandro De Luca //
	Proceedings of the 2000 IEEE International Conference on Robotics and Automation. San Francisco. CA.
	2000. P. 233-240.
	
	\bibitem{luka05} 
	{\it Alessandro De Luca} PD control with on-line gravity compensation for robots with elastic joints: Theory and experiments / Alessandro De 		Luca, Bruno Siciliano, Loredana Zollo. // Automatica. 2005. Vol. 41. P. 1809-1819.
	
	\bibitem{alonge03}
	{\it Alonge F.} An adaptive control law for robotic manipulator without velocity feedback / Alonge F., D’Ippolito F., Raimondi F.M. // Control 		Engineering Practice. 2003. 11. P. 999-1005.

	\bibitem{andreev16}
	{\it Andreev A.} Motion Control of Multilink Manipulators without Velocity Measurement / A. S. Andreev, O. A. Peregudova, D.S. Makarov // OASTAB 	- 16
	
	\bibitem{araki71}
	{\it Araki, M.} Stability of sampled-data composite systems with many nonlinearities / M. Araki, K. Ando, B. Kondo // IEEE Trans. Automat. 		Contr.— AC-16, 1971.— Pp. 22–27.
	
	\bibitem{arimoto85}
	{\it Arimoto S., Miyazaki F.} Asymptotic stability of feedback control laws for robot manipulators // Symp. on robot control 85, Barcelona? 6-8 Nov/ 1985.

	\bibitem{arimoto95}
	{\it Arimoto S., Naviwa T., Nakayama T., Parra-Vega V., Liu Y.H.} A hyper-stability theoretic consideration on coordinated control of multiple robot arms // Proc. of 3rd European Control Conference. Rome. Italy. September 1995. P. 1894-1899. 
	
	\bibitem{balafoutis}
	{\it Balafoutis C.} Efficient modeling and computation of manipulator dynamics using orthogonal cartesian tensors. / Balafoutis C., Patel R., 		Misra P. // IEEE J. of Rob. and Autom., 4, N 6, pp.665-676.
	
	\bibitem{balestrino}
	{\it Balestrino A., De Maria G., Sciavicco L.} An adaptice model following control system for robotic manipulators // ASME J. Dynamic Systems. Measurement and Control. 1983. V. 105, № 3.
	
	\bibitem{berghuis91}
	{\it Berghuis H.} Tracking control of robots using only position measurements / Berghuis H., Lѐohnberg P., Nijmeijer H. // 30th Conf. on 		Decision and Control. 1991. Vol. 1. P. 1039-1040.
	
	\bibitem{berghuis93_2}
	{\it Berghuis H.} A passivity approach to controller-observer design for robots / Berghuis H. and Nijmeijer H. // IEEE Transactions on robotics 	and automation. 1993. Vol. 9. No 6. P. 740-754.

	\bibitem{berghuis93_1}
	{\it Berghuis H.} Global regulation of robots using only position measurements / Berghuis H. and Nijmeijer H. // Systems and Contr. Letters, 		vol. 21, 1993, pp. 289–293. 
	
	\bibitem{brogliato95}
	{\it Brogliato B.} Global tracking controllers for flexible-joint manipulators: a comparative study / Brogliato B., Ortega R., Lozano R. // 		Automatica. 1995. Vol. 31. No 7. P. 941-956.
	
	\bibitem{burkov95}
	{\it Burkov I.V.} Stabilization of mechanical systems via bounded control and without velocity measurement / Burkov I.V. // 2nd Russian-Swedish 	Control Conf. St. Petersburg: St. Petersburg Technical Univ. 1995. P. 37-41.
	
	\bibitem{burkov952}
	{\it Burkov I.V., Freidovich L.B.} Stabilization of the program position of elastic robot via control with saturation // Proc. of 10-th IFAC Workshop on Control Appl. of Optimization. 19-21 December 1995. Haifa. Izrael.
	%\bibitem{d301}
	%{\it Bogdanov A. Yu.} The Discrete Nonlinear Nostationary Lossless Systems: Stabilization and Feedback Equivalence // Dynamic Systems and 		Applications.— Vol. 3.- Dynamic Publishers, Inc, USA.- 2001.— Pp. 91–98.
	
	\bibitem{burkov09}
	{\it Burkov I.V.} Stabilization of position of uniform motion of mechanical systems via bounded
	control and without velocity measurements / Burkov I.V. // 3rd IEEE Multti - conference on Systems and Control.
	St. Petersburg. 2009. P.400-405.
	
	\bibitem{calugi02}
	{\it Calugi F.} Output feedback adaptive control of robot manipulators using observer backstepping / Calugi F., Robertsson A., Johansson R. // 		Proceedings of the 2002 IEEE/RSJ Int. Conference of Intelligent Robots and Systems. Lausanne, Switzerland. 2002. P.2091-2096.
	
	\bibitem{dixon04}
	{\it Dixon W.E.} Global robust output feedback tracking control of robot manipulators / Dixon W.E., Zergeroglu E. and Dawson D.M. // Robotica. 		2004. Vol. 22. P. 352-357.
	
	\bibitem{kelly93}
	{\it Kelly R.} A simple set-point robot controller by using only position measurements / Kelly R. // In Preprint 12th IFAC World Congress, vol. 	6, Sydney, 1993, pp. 173–176.
	
	\bibitem{lindtner88}
	{\it Lindtner E.} Nichtlineare Stabilitatsuntersuchungen eines DD-Manipulators / Lindtner E., Steindl A., Troger H. // Z. angew. Math. und Mech., 	1988, 68, № 4, 75 – 77 – РЖМех, 1988, 10А74.
	
	\bibitem{loria96}
	{\it Loria A.} Global tracking control of one degree of freedom Euler-Lagrange systems without velocity measurements / Loria A. // European J. 		Contr., vol. 2, 1996, pp. 144–151.
	
	\bibitem{loria98}
	{\it Loria A.} Loria A., Nijmeijer H. Bounded output feedback tracking control of fully actuated Euler–Lagrange systems // Systems and Control 		Letters. 1998. 33 (3). P. 151-161.

	\bibitem{loria13} 
	{\it Loria A.} Observer-less output feedback global tracking control of lossless Lagrangian systems / Loria A. // arXiv preprint arXiv:			1307.4659, 2013/7/17.

	\bibitem{loria95}
	{\it Loria A.} On tracking control of rigid and flexible joints robots / Loria A., Ortega R. // Appl. Math. Comput. Sci. 5(2), 1995, pp.101-113.
	
	\bibitem{mahil82}
	{\it Mahil S.} On the application of Lagrange's method to the description of dynamic systems. / Mahil S. // IEEE Trans. on SMC, vol SMC-12, N 6, 		1982.
	
	\bibitem{moberg07} 
	{\it Moberg S.} On modeling and control of flexible manipulators. / Moberg S. // Linkoping: Linkoping University,
	2007. 148 p.
	
	\bibitem{natvani01} 
	{\it Natvani L.} On the asymptotic stability for functional differential equations by Lyapunov functionals  / Natvani L. // Nonlinear Anal., 		Theory Methods Appl. – 2001. – V. 47, No. 7. – P . 4333 – 4343. – P. 315 – 323.
	
	\bibitem{navrol992}
	{\it Navarro-Lopez E. M.} Dissipativity, passivity and feedback passivity in the nonlinear discrete-time setting. / E. M. Navarro-Lopez, E. 		Fossas-Colet.- 1999.
		
	\bibitem{nicosia84}
	{\it Nicosia S.} Nicosia S., Tomei P. Model reference adaptive control algorithms for industrial robots // Automatica, 20, 1984.	
	
	\bibitem{nicosia90}
	{\it Nicosia S.} Nicosia S., Tomei P. Robot control by using only joint position measurements // IEEE Trans. Aut. Contr. 1990. V. 35, № 9. P. 		1058-1061.

	\bibitem{nunes08}
	{\it Nunes E.} Arbitrarily small damping allows global output feedback tracking of a class of Euler-Lagrange systems / Eduardo V. L. Nunes, Liu 	Hsu and Fernando Lizarralde. // American Control Conference Westin Seattle Hotel, Seattle, Washington, USA, June 11-13, 2008. P. 377-382.
	
	\bibitem{olsen}
	{\it Olsen H.B., Bekey G.A.} Identification on robot dynamics, in Proc. IEEE Conf. on Robotics and Automation, 1984.
	
	\bibitem{sell67}
	{\it Sell G. R.} Nonautonomous differential equations and topological dynamics 1, 2 /
	G. R. Sell // Trans. Amer. Math. Soc.— 1967.— Vol. 127.— Pp. 241–283.
	
	\bibitem{takegaki} 
	{\it Takegaki M., Arimoto S.} A new feedback method for dynamic control of manipulators // ASME Trans., J. Dynam. Syst. Measure. Control. 1981.
	
	\bibitem{togia} 
	{\it Togia M., Yamano O.} Learning control of robot manipulators. SLAM Conf. Geometric modeling and robotics, Albany, NY, July 15-17, 1985.
	
	\bibitem{tomei91} 
	{\it Tomei, P.} A simple PD controller for robots with elastic joints / Tomei, P. // IEEE Transactions on
	Automatic Control. 1991. 36(10). P. 1208-1213.
	
	\bibitem{utkin92} 
	{\it Utkin V.I.} Sliding Models in Optimization and Control. -- New York: Springer-Verlang, 1992.
	
	\bibitem{vukobratovic79}
	{\it Vukobratovic M.} Contribution to automatic forming of active chain models via Lagrangian form. / Vukobratovic M., Potkonjak V. //J of Appl. 	Mech., N 1, 1979.
	
	\bibitem{vukob85} 
	{\it Vukobratovic M.} Scientific fundamentals of robotics, control of manipulations robots: theory and application. / Vukobratovic M., Stokic 		D. // – Berlin: Springer-Verlang, 1982. Перевод: Вукобратович М.К., Стокич Д.М. Управление манипуляционными роботами: Теория и применение. – М.: 	Наука, 1985. – 384 с.
	
	\bibitem{vukob89} 
	{\it Vukobratovic M.} Non-adaptive and adaptive control of manipulation robots. / Vukobratovic M., Stokic D., Kircanski N. // – Berlin: 			Springer-Verlag, 1985. Перевод: Вукобратович М., Стокич Д., Кирчански Н., Неадаптивное и адаптивное управление манипуляционными роботами. – М.: 	Мир, 1989ю – 376 с.
	
	\bibitem{yarza11}
	{\it Yarza A.} Uniform Global Asymptotic Stability of an Adaptive Output Feedback Tracking Controller for Robot Manipulators / Yarza A., 		Santibanez V., Moreno-Valenzuela J. // Preprints of the 18th IFAC World Congress Milano (Italy) August 28 - September 2, 2011. P. 14590-14595. 
	
	
\end{thebibliography} 