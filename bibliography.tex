\begin{thebibliography}{10} \label{bibl}
	\bibtem{kuleshov}
	Кулешов В.С., Лакота Н.А. Динамика систем управления манипуляторами. - М.: Энергия, 1971. -304 с.
	
	\bibtem{ignatev}
	Игнатьев М.Б., Кулаков Ф.М., Покровский А.М. Алгоритмы управления роботами-манипуляторами. - Л.: Машиностроение, 1972. - 360 с.
	
	\bibtem{pol}
	Пол Р. Моделирование, планирование траекторий и управление движением робота-манипулятора/Перевод с англ. - М.: Наука. 1976. - 103 с.
	
	\bibtem{popov1}
	Попов Э.В., Фридман Г.Р. Алгоритмические основы интеллектуальных роботов и искусственного интеллекта. - М.: Наука, 1976. - 4S6 с.
	
	\bibtem{popov2}
	Попов Е.П., Верещагин А.Ф. Зенкевич С.Л., Манипуляционные роботы. Динамика и алгоритмы. - М.: Наука. 1978. - 398 с.
	
	\bibtem{medvedev}
	Медведев Н.С., Лесков А.Г., Ющенко А.С. Системы управления манипуляционных роботов. - М.: Наука. 1978. - 416 с.
	
	\bibtem{frolov1}
	Фролов К.В., Воробьев Е.И. Механика промышленных роботов. Кн. 1. Кинематика и динамика. - М.: Высш. шк., 1988. - 304 с.
	
	\bibtem{frolov2}
	Фролов К.В., Воробьев Е.И. (ред.) Механика промышленных роботов. Кн. 2. Расчет и проектирование механизмов. - М.: Высш. шк., 1988. - 367 с.
	
	\bibtem{frolov3}
	Фролов К.В., Воробьев Е.И. (ред.) Механика промышленных роботов. Кн. 3. Основы конструирования. - М.: Высш. шк., 1988. - 383 с.
	
	\bibtem{korend}
	Корендясев А.И., Саламандра Б.Л., Тывес Л.И. Теоретические основы робототехники. В 2 кн. Кн. 2. - М.: Наука, 2006. - 383 с.
	
	\bibtem{zenkevich}
	Зенкевич С.Л., Ющенко А.С. Управление роботами. Основы управления манипуляционными роботами: Учеб. для вузов - М.: МГТУ им. Н.Э.Баумана, 2000. - 400 с
	
	\bibtem{urevich}
	Юрееич Е.И. Основы робототехники. 2-е изд., переизд. и доп. - СПб.: БХВ-Петербург, 2005. - 416 с.
	
	\bibtem{balakireva}
	Балакирева Т.Н., Воробьев Е.И. Механика роботов (кинематика, динамика, синтез механизмов), «Итоги науки к техники». ВИНИТИ. - Сер. «Общая механика». 1990. Вып. 7. - С. 3-89.
	
	\bibtem{krutko1}
	Крутько П.Д. Управление исполнительными системами роботов. - М.: Наука, 1991.
	
	\bibtem{krutko2}
	Крутько П.Д. Управление движением лагранжевых систем. Синтез алгоритмов методом обратных задач динамики // ТСУ. 1995. № 6. С. 19-37.
	
	\bibtem{krutko3}
	Крутько П.Д. Управление движением Эйлеровских систем. Синтез алгоритмов методом обратных задач динамики // ТСУ. 1995. № 1. С. 34-53.
	
	\bibtem{krutko4}
	Крутько П.Д., Лакота Н.А. Метод обратных задач динамики в теории конструирования алгоритмов управления манипуляционных роботов. Задача стабилизации // Изв. АН СССР. Техническая кибернетика. 1987. № 3,4.
	
	\bibtem{krutko5}
	Крутько П.Д., Попов Е.П. Кинематические алгоритмы управления движением манипуляционных роботов // Изв. АН СССР. Техническая кибернетика. 1979. № 4. С. 77-86.
	
	\bibtem{galiull}
	Галиуллин А.С. Методы решения обратных задач динамики. - М. Наука, 1986. - 224 с. 
	
	\bibtem{petrov}
	Петров Б.Н., Крутько П.Д., Попов Е.П. Построение алгоритмов управления как обратная задача динамики // Докл. АН СССР. 1979. № 5. С. 1078-1081.
	
	\bibtem{formal}
	Формальский А.М. Управляемость и устойчивость систем с ограниченными ресурсами. - М.: Наука, 1974. - 367 с.
	
	\bibtem{golubev}
	Голубев А.Е. Стабилизация нелинейных динамических систем с использованием оценки состояния системы асимптотическим наблюдателем (Обзор) / А.Е. Голубев, А.П. Крищенко, С.Б. Ткачев / / Автоматика и телемеханика. - 2005. - №7. - С. 3 - 42.
	
	\bibtem{lapunov}
	Ляпунов А.М. Избранные труды. Работы по теории устойчивости / А.М. Ляпунов - М Наука, 2007.-574 с.
	
	\bibtem{rumancev1}
	Румянцев В.В. Устойчивость и стабилизация движения по отношению к части переменных / В.В. Румянцев, А.С. Озиранер - М.: Наука, 1987. - 253 с.
	
	\bibtem{krasovskii}
	Красовский Н.Н. Некоторые задачи теории устойчивости движения / Н.Н. Красовский - М.> Физматгиз, 1959. - 211 с.
	
	\bibtem{rush}
	Руш Н. Прямой метод Ляпунова в теории устойчивости / Н. Руш, П. Абетс, М. Лалуа - М.: Мир, 1980. - 300 с.
	
	\bibtem{andreev}
	Андреев А.С. О стабилизации движения нестационарной управляемой системы / А.С. Андреев, В.В. Румянцев // Автоматика и телемеханика. - 2007. -  № 8. - С. 18—31.
	
	\bibtem{rumancev2}
	Румянцев В.В. О стабилизации движения нестационарной управляемой системы / B.В Румянцев. А.С. Андреев // Доклады Академии наук. - 2007. - Т. 416, №5. - С. 627—629.
	
	\bibtem{halil}
	Халил К.К. Нелинейные системы / Х.К. Халил - М.-Ижевск: НИЦ "Регулярная и хаотическая динамика", Институт компьютерных исследований, 2009. - 832 с.
	
	\bibitem{and79}
	Андреев А.С. Об асимптотической устойчивости и неустойчивости неавтономных систем // ПММ. 1979. Т.43. Вып.5. С.796--805.
	
	\bibitem{and841}
	Андреев А.С. Об асимптотической устойчивости и неустойчивости нулевого решения неавтономной системы // ПММ. 1984. Т.48. Вып.2.
	
	\bibitem{and87}
	Андреев А.С. Об исследовании частичной асимптотической устойчивости и неустойчивости на основе предельных уравнений // ПММ. 1987. Т.51. Вып.2. C.253--260.
	
	\bibitem{and912}
	Андреев А.С. Об исследовании частичной асимптотической устойчивости // ПММ. 1991. Т.55.
	Вып.4. C.539--547.
	
	\bibitem{afanasev}
	Афанасьев В.Н. Математическая теория конструирования систем управления: Учеб. для вузов / B.Н Афанасьев, В.Б. Колмановскй, В.Р. Носов - 3-е изд., испр. и доп. М.: Высш. шк., 2003. -- 614 с.
	
	\bibitem{roitenberg}
	Ройтенберг Я.Н. Автоматическое управление / Я.Н. Ройтенберг - М.: Наука, 1971. -- 395 с.
	
	\bibitem{dorf}
	Дорф Р. Современные системы управления / Р. Дорф, Р. Бишоп; Пер. с англ. Б.И. Копылова - М.: Лаборатория Базовых знаний, 2004. - 832 с.
	
	\bibitem{malkin}
	Малкин И.Г. Теория устойчивости движения / И.Г. Малкин — М.: Наука, 1966. - 530 с.
	
	\bibitem{aleksandrov}
	Александров В.В. Оптимизация динамики управляемых систем / В.В. Александров, B.Г. Болтянский, С.С. Лемак, А. Парусников, В.М. Тихомиров - М.: МГУ, 2000. - 303 с.
	
	\bibitem{andb02}
	Андреев А.С., Бойкова Т.А. Знакопостоянные функции Ляпунова в задачах об устойчивости // Механика твердого тела. 2002. Вып.32. С.109--116.
	
	\bibitem{andb06}
	Андреев А.С. К методу сравнения в задачах об асимптотической устойчивости / А.С. м Андреев, О.А. Перегудова // ПММ. - 2006. - Т. 70, Вып. 6. - С. 965—976.
	
	\bibitem{pereg06}
	Перегудова О.А. Функции Ляпунова вида векторных норм в задачах устойчивости / О.А. Перегудова // Уч. зал. Ульян, гос. ун-та. Сер. «Фундаментальные проблемы математики и механики». - Ульяновск: Изд-во УлГУ, 2006. - Вып. 1(16). - С. 43	51.
	
	\bibitem{rum1}
	Румянцев В.В. О стабилизации движения нестационарной управляемой cистемы / В.В Румянцев, А.С. Андреев // Доклады Академии наук. - 2007. - Т. 416, № 5. - С. 627—629.
	
	\bibitem{pereg08}
	Перегудова О.А. Логарифмические матричные нормы в задачах устойчивости движения / О.А. Перегудова // ПММ. - 2008. - Т. 72, Вып. 3. - С. 410—420.
	
	\bibitem{heil84}
	Хейл Дж. Теория функционально-дифференциальных уравнений / Дж. Хейл - М.: Мир, 1984 - 422 с.
	
	\bibitem{kolman81}
	Колмановский В.Б. Устойчивость и периодические режимы регулируемых систем с последействием / В.Б. Колмановский, В.Р. Носов - М.: Наука, 1981. - 448 с.
	
	\bibitem{and05}
	Андреев. А.С. Устойчивость неавтономных функционально-дифференциальных уравнений / А.С. Андреев. — Ульяновск: УлГУ, 2005. — 328 с.
	
	\bibitem{and09}
	Андреев А.С. Метод функционалов Ляпунова в задаче об устойчивости функционально-дифференциальных уравнений / А.С. Андреев // Автоматика и телемеханика. - 2009, № 9. - С. 4-55.
	
	\bibitem{krasov63}
	Красовский Н.Н. О стабилизации движений управляемого объекта с запаздыванием в системе регулирования / Н.Н. Красовский, Ю.С. Осипов // Известия АН СССР. ТК. - 1963. - № 6. 3—15.
	
	\bibitem{kim96}
	Ким А.В. i-Гладкий анализ и функционально-дифференциальные уравнения / А.В. Ким - Екатеринбург: УрО РАН, 1996. - 233 с.
	
	\bibitem{pavl06}
	Павликов С.В. Метод функционалов Ляпунова в задачах устойчивости / С.В. Павликов Набережные Челны: Ин-т управления, 2006.
	
	\bibitem{and97}
	Андреев А.С. Об устойчивости неавтономного функционально дифференциального уравнения // Доклады РАH. 1997. Т.356. \No 7. C.151--153.
	
	\bibitem{routh77}
	Routh E.J. A Treatise on the Stability of a Given State of Motion. -- London: MacMillan and Co., 1877.
	
	\bibitem{rumancev66}
	Румянцев В.В. Об устойчивости стационарных движений // ПММ. -- 1966. Т. 30. -- С. 922-923.
	
	\bibitem{karapetian98}
	Карапетян А.В. Устойчивость стационарных движений. -- М.: УРСС, 1998.
	
	\bibitem{volterra82}
	Вольтерра В. Теория функционалов, интегральных и интегродифференциальных уравнений. -- М.: Наука, 1982. -- 304 с.
	
	\bibitem{markeev99}
	Маркеев А.П. Теоретическая механика / А.П. Маркеев — М.: ЧеРо, 1999. - 569 с.
	
	\bibitem{merkin74}
	Меркин Д.Р. Гироскопические системы / Д.Р. Меркин — М.: Наука, 1974. -- 344 с.
	
	\bibitem{merkin03}
	Меркин Д.Р. Введение в теорию устойчивости движения / Д.Р. Меркин — 4-е изд., стер, - СПб.: Лань, 2003. -- 304 с.
	
	\bibitem{andr04}
	Андреев А.С. Об устойчивости неустановившегося движения механической системы / А.С. Андреев, Т.А. Бойкова // ПММ. - 2004. - Т. 68, Вып. 4. - С. 678—686.
	
	\bibitem{kalen10}
	Каленова В.И., Морозов В.М. Линейные нестационарные системы и их приложения к задачам механики // М.: ФИЗМАТЛИТ, 2010 - 208 с.
	
	\bibitem{pereg07}
	Перегудова О.А. Уравнения сравнения в задачах об устойчивости движения // Автоматика и телемеханика. 2007. №9. С.56-63.
	
	\bibitem{karapetian83}
	Карапетян А.В. Устойчивость консервативных и диссипативных систем / А.В. Карапетян, В.В. Румянцев - М.: ВИНИТИ, 1983. - (Итоги науки и техники. Сер. Общая механика; Т.6)
	
	\bibitem{chernous92}
	Черноусько Ф.Л. Синтез управления нелинейной динамической системой / Ф.Л. Черноусько // ПММ. - 1992. - Т. 56, Вып. 2. - С. 179—191.
	
	\bibitem{anan95}
	Ананьевский И.М. Метод декомпозиции в задаче управления механической системой / И.М. Ананьевский, И.С. Добрынина, Ф.Л. Черноусько // Известия РАН. Теория и системы управления. - 1995.- №2. - С. 3—14.
	
	\bibitem{chernous06}
	Черноусько ФЛ. Методы управления нелинейными механическими системами / Ф.Л. Черноусько, И.М. Ананьевский, С.А. Решмин. - М.: ФИЗМАТЛИТ, 2006. - 328 с.
	
	\bibitem{patnic87}
	Пятницкий Е.С. Синтез управления манипуляционными роботами на принципе декомпозиции / Е.С. Пятницкий // Известия АН СССР. Техническая кибернетика. - 1987. - № 3. - С. 92—99.
	
	\bibitem{matuh891}
	Матюхин В.И., Пятницкий Е.С. Синтез систем управления многозвенными механизмами на принципе декомпозиции при учете динамики приводов // Машиноведение. 1989. № 3. С. 42-48.
	
	\bibitem{matuh892}
	Матюхин В.И., Пятницкий Е.С. Управление движением манипуляционных роботов на принципе декомпозиции при учете динамики приводов // Автоматика и телемеханика. 1989. № 9. С. 67-82.
	
	\bibitem{matuh04}
	Матюхин В.И. Пятницкий Е.С. Управляемость механических систем в классе управлений, ограниченных вместе с производной // Автоматика и телемеханика. 2004. № 8. С. 14-38.
	
	\bibitem{matuh93}
	Матюхин В.И. Устойчивость движения механических систем при учете постоянно действующих возмущений // Автоматика и телемеханика. 1993. № 11. С. 124-134.
	
	\bibitem{matuh01}
	Матюхин В.И. Универсальные законы управления механическими системами. - М.: МАКС Пресс, 2001. 249 с.
	
	\bibitem{pereg09}
	Перегудова О.А. К задаче слежения для механических систем с запаздыванием в управлении / О.А. Перегудова // Автоматика и телемеханика. — 2009. —  № 5. - С. 95—105.
	
	\bibitem{pavlik071}
	Павликов С.В. О стабилизации движений управляемых систем с запаздывающим регулятором / С.В. Павликов // Доклады Академии наук. - 2007. - Т. 412, № 2. — С. 176—178.
	
	\bibitem{pavlik072}
	Павликов С.В. К задаче о стабилизации управляемых механических систем / С.В. Павликов // Автоматика и телемеханика. - 2007. - № 9. - С. 16—26.
	
	\bibitem{andp991}
	Андреев А.С., Павликов С.В. Об устойчивости по части переменных неавтономного функционально дифференциального уравнения // ПММ. 1999. Т.63. Вып.1. С. 3--12.
	
	\bibitem{and98}
	Андреев А.С., Хусанов Д.Х. Предельные уравнения в задаче об устойчивости функционально-дифференциального уравнения // Диффер. уравнения. - 1998. - Т. 34, № 4. - С. 435-440.
	
	\bibitem{and12}
	Андреев А.С. Об управлении движением голономной механической системой / А.С. Андреев, А.О. Артемова // Научно-технический вестник Поволжья. -— 2012. — № 6.
	
	\bibitem{artem12}
	Артемова А.О. Моделирование управляемого движения двузвенного манипулятора на подвижном основании / А.О. Артемова // Научно-технический вестник Поволжья. — 2012. — № 6.
	
	\bibitem{tadziev12}
	Таджиев Д.А. Об управлении движением голономной системы с учетом динамики приводов / Д.А. Таджиев / / Научно-технический вестник Поволжья. — 2012.— № 6.
	
	\bibitem{and13}
	Андреев А.С., Благодатнов В.В., Кильметова А.Р. Уравнения Вольтерра в моделировании ПИ- и ПИД-регуляторов // Научно-технический вестник Поволжья. - 2013. - № 1. - С. 84-90.
	
	\bibitem{andtm13}
	Андреев А.С., Макаров Д.С., Таджиев Д.А. Об управлении двузвенным манипулятором с приводом // Научно-технический вестник Поволжья, № 5,2013. С. 102-105.
	
	\bibitem{vukobrat851}
	Vukobratovic M., Stokic D. Scientific fundamentals of robotics, control of manipulations robots: theory and application. - Berlin: Springer-Verlag, 1982. Перевод: Вукобратович М.К., Стокич Д.М. Управление манипуляционными роботами: Теория и применение. - М.: Наука, 1985. - 384 с.
	
	\bibitem{vukobrat852}
	Vukobratovic M., Stokic D., Kircanski N. Non-adaptive and adaptive control of manipulation robots. - Berlin: Springer-Verlag, 1985. Перевод: Вукобратович М., Стокич Д., Кирчански Н., Неадаптивное и адаптивное управление манипуляционными роботами. - М.: Мир, 1989. - 376 с.
	
	\bibitem{chernous84}
	Черноусько Ф.Л., Болотник Н.Н., Градецкий В.Г. Манипуляционные роботы: динамика, управление, оптимизация. - М.: Наука, 1984.
	
	\bibitem{gradec10}
	Градецкий В.Г., Князьков М.М., Фомин Л.Ф., Чащухин В.Г., Механика миниатюрных роботов. - М.: Наука, 2010. - 271 с.
	
	\bibitem{wittenburg80}
	Wittenburg J. Dynamics of systems rigid bodies. - Stuttgart: B.G. Tubner, 1977. Перевод: Виттенбург И. Динамика систем твердых тел. - М.: Наука, 1980. - 262 с.
	
	\bibitem{sell67}
	Sell G.R. Nonautonomous differential equations and topological dynamics. 1, 2 / G.R.Sell // Trans. Amer. Math. Soc., 1967. - V. 22. - P. 254--269.
	
	\bibitem{artstein77}
	Artstein Z. Topological dynamics of ordinary differential equations / Z. Artstein // J. Differ. Equat. - 1977. - V. 23, № 2. - P. 216-223.
	
	\bibitem{and06}
	Андреев А.С. Знакопостоянные функции Ляпунова в задачах устойчивости / А.С. Андреев, Т.А. Бойкова // Механика твёрдого тела. Донецк: Ин-т прикл. мат. и мех. - 2002. - Вып. 32. - С. 109--116.
	
	\bibitem{and062}
	Андреев А.С. К методу сравнения в задачах об ассимптотической устойчивости / А.С. Андреев, О.А. Перегудова // ПММ. - 2006. - Т. 70, Вып. 6. - С. 965--976.
	
	\bibitem{andr97}
	Andreev A. On the stability of nonautonomous functional differential equations / A. Andreev // Nonlinear Anal. TMA. - 1997. - V. 30, Part 5. - P. 2847--2854.
	
	\bibitem{andr98}
	Андреев А.С. К методу функционалов Ляпунова в задаче об асимптотической устойчивости и неустойчивости / А.С. Андреев, Д.Х. Хусанов // Дифференц. уравнения. - 1998. - Т. 34, № 7. - С. 876-885. 
	
	\bibitem{pereg091}
	Перегудова О.А. Метод сравнения в задачах устойчивости и управления движениями механических систем / О.А. Перегудова - Ульяновск: Изд-во УлГУ, 2009. -- 253 с.
	
	\bibitem{andr09}
	Андреев А.С. О стабилизации движений механических систем с переменными массами / А.С. Андреев, Р.Б. Зайнетдинов // ПММ. - 2009ю - Т. 73, Вып. 1. - С. 3--12.
	
	\bibitem{pereg092}
	Перегудова О.А. О стабилизации движений неавтономных механических систем / О.А. Перегудова // ПММ. - 2009ю - Т. 73, Вып. 2. - С. 176--188.
	
	
\end{thebibliography} 