\section{Стабилизация программных позиций голономной мехнанической системы без измерения скоростей} \label{p12}

Движение управляемой механической системы с стационарными голономными связями, имеющей $n$ обобщенных координат $q_1, q_2, ... q_n$ может быть описано уравнениями Лагранжа 

\begin{equation}
\frac{d}{dt} (\frac{\partial T}{\partial \dot q}) - \frac{\partial T}{\partial q} = Q + U
\end{equation}

 В уравнениях введена векторно-матричная запись: $q \in R^n, q = (q_1, q_2, ... q_n)^{'}$ --- вектор обобщенных координат, $\dot q = (\dot q_1, \dot q_2, ... \dot q_n)^{'}$ --- вектор обобщенных скоростей. $T = \frac{1}{2} (\dot q)^{'} A(q) \dot q$ --- кинетическая энергия с инерциальной матрицей $A(q), A \in R^{n \times n};  Q = Q(t, q, \dot q)$ --- вектор обобщенных сил, которые полагаются зависимыми от $(t, q, \dot q);$ $U$ - обобщенная управляющая сила; через $( )^{'}$ обозначена операция транспонирования. 

В дальнейшем обозначим через $\| q \|$ норму вектора $q$, $\|q\|^{2} = q_1^2 + q_2^2 + ... + q_n^2.$

Инерционная матрица $A$ в общем случае является определенно-положительной при всех фиксированных допустимых значениях обобщенных координат. Будем полагать, что система (2.1) определяется при всех $q \in R^n,$ при этом соответсвующая квадратичная форма $(\dot q)^{'} A(q) \dot q$ является ограниченной определенно-положительной при всех $q \in R^n,$ так что имеет место оценка 

\begin{equation}
2 \alpha_0 \| \dot q \|^2 \le (\dot q)^{'} A(q) \dot q \le 2 \alpha_1 \| \dot q \|^2, \alpha_0 > 0
\end{equation}

При этом полагаем, что $A(q)$ непрерывно-дифференцируема.

В моделировании манипуляторов представляется важным учет потенциальных гироскопических и диссипативных сил. Соответственно будем полагать, что обобщенные силы непрерывны по $(t, q, \dot q) \in R^{+} \times R^{n} \times R^{n},$ и при этом представимы в виде

\begin{equation}
Q(t, q, \dot q) = -\frac{\partial \Pi (t, q)}{\partial q} + Q_g (t, q, \dot q)
\end{equation}

 где $\Pi = \Pi(t, q)$ --- потенциальная энергия, $Q_g(t, q, \dot q)$ - совокупность диссипативных и гироскопических сил. По их определению для $Q_g$ будут иметь место следующие соотношения
 
 \begin{equation}
   Q_g \equiv 0, \dot q^{'} Q_g (t, q, \dot q) \le 0 \forall \dot q \in R^n
 \end{equation}
 
Задача о стабилизации программной позиции системы (2.1) состоит в нахождении управления $U,$ обеспечивающего стабилизацию положения системы 

\begin{equation}
\dot q = 0, q = q_0 = const
\end{equation}

Для решения этой задачи могут быть использованы традиционные методы исследования асимптотической устойчивости под действием структуры сил и их развитие []. Эти методы предполагают наличие диссипативных сил с полной, а иногда и с частичной диссипацией. Таким образом для построения управляющих сил требуется измерение всех (и лишь иногда части) обобщенных скоростей. 

Использование теорем об устойчивости фунционально-дифференциальных уравнений, представленое в параграфе 1.1 позволяет решить эту задачу только посредством измерения обобщенных координат.

Уравнения движения (2.1) с учетом представленных обобщенных сил (2.3) могут быть записаны в виде 

 \begin{equation}
 A(q) \ddot q = C(q, \dot q) \dot q - \frac{\partial \Pi(t, q)}{\partial q} + Q_g(t, q, \dot q) + U
 \end{equation}

 где матрица $C = (c_{j,k})$ инерционных сил определяется равенством $c_{j,k} = \frac12 \sum_{i =1}^{n} (\frac{\partial a_{ik} } {\partial q_j} - \frac{\partial a_{kj}}{\partial q_i} - \frac{\partial a_{ij}}{\partial q_k}) \dot q_i$

Пусть (2.5) есть некоторая выбранная программная позиция. Введем возмущение
\begin{equation}
x = q - q_0, y = \dot x = \dot q
\end{equation}
 
Уравнения движения (2.6) в переменных $(x, \dot x)$ принимают следующий вид 

\begin{equation}
\frac{dx}{dt} = y, \frac{dy}{dt} = A^{-1}_1(x) (C_1(x, y) y - \frac{\partial \Pi_1 (t, x)}{\partial x} + Q_1 (t, x, y) + U)
\end{equation}

где индексом "$_1$" обозначены функции, получаемые из соответствующих зависимостей (2.6) в результате замены (2.7).

Введем в рассмотрение матрицы $P = P(t, \tau), P \in R^{n \times n},$ учитывающие предыдущее состояние системы в виде зависимостей $P = \| p_{jk} (t, \tau) \|, p_{jk} = p_{jk}^0 e^{S_{jk}^0 (\tau - t)}$ с постоянными $p_{jk}^0 = const, s_{jk}^0 = const, p_{jk}^0 = p_{kj}^0; s_{jk}^0 = s_{kj}^0$ такими, что выполнены условия


\begin{equation}
 x^{'} P (t, \tau) x \ge 0; x^{'} P_t (t, \tau) x \le - \beta_0 \| x \|^2, \beta > 0 
  P_t(t, \tau) = \frac{\partial P (t, \tau)}{\partial t} = \\ = - \| p_{jk}^0 s_{jk}^{0} e^{S_{jk}^0 (\tau - t)} \|
\end{equation}

Покажем, что поставленная задача может быть решена управлением вида
\begin{equation}
U = - \frac{\partial \Pi (t, x(t))}{\partial x} - \int_{t - h}^{t} P(t, \tau) (x(t) - x(\tau)) d \tau
\end{equation}

Уравнения (2.8) при управлении (2.10) принимают следующий вид

\begin{equation}
\frac{d x(t)}{dt}=y(t), \frac{d y(t)}{dt} = A_1^{-1} (x(t)) (C_1(x(t), y(t)) + Q_1(t, x(t), y(t)) - \frac{\partial \Pi_0 (t, x)}{\partial x} - \int_{t - h}^{t} P(t, \tau) (x(t) - x(\tau)) d \tau)
\Pi_0 (t, x) = \Pi_1(t, x) + \Pi_k (t, x), \Pi_0 (t, 0) \equiv 0
\end{equation}

Это уравнение представляет собой совокупность функционально-дифференциальных уравнений с запаздывающим аргументом с конечным запаздыванием. Областью определения этого уравнения можно принять область вида $R \times C_1 \times C_2,$ где $C_1$ --- пространство непрерывных функций $\varphi : [-h, 0] \to R^n$ с нормой $\| \varphi \|_c = \sup (\| \varphi(s) \|, -h \le s \le 0), C_2$ --- пространство непрерывных функций $\psi : [-h, 0] \to R^n$ с аналогичной нормой.

Для движения $(q(t), \dot q(t))$ системы (1.1) с начальным состоянием $(q(t_0), \dot q(t_0))$ за начальную функцию системы уравнений (2.11) можно принять функцию $\varphi (s) = x(t_0) = q(t_0) - q_0, \psi (s) = y(t_0) = \dot q (t_0), -h \le s \le 0.$

Функции $A_1^{-1} (x(t))$ и $C_1(x(t), y(t))$ представляют собой равномерно непрерывные функции по отношению к непрерывной совокупности $(x(t), y(t)), \| x(t) \| + \| y(t) \| \le H, H = const > 0,$ на некотором конечном отрезке $[t_0, t_0 + T].$

Будем полагать, что зависимости $Q = Q(t, x, y), \Pi_0 (t, x), \frac{\partial \Pi_0 (t, x)}{\partial x}$ представляют собой функции, ограниченные, равномерно непрерывные по $(t, x, y)$ на каждом множестве вида $R \times \lbrace \| x \| + \| y \| \le H \rbrace$ при любом $H > 0.$

Соответственно согласно п 1.1. и работе [] семейства предельных функций $\lbrace Q^{*} (t, x, y) \rbrace$ и $\lbrace \Pi_0^{*} (t, x) \rbrace$ определяются равенствами $Q^{*} (t, x, y) = \lim_{t_k \to \infty} Q(t_k + t, x, y), \Pi_0^{*} (t, x) = \lim_{t_k \to \infty} \Pi_0 (t_k + t, x) \frac{\partial \Pi_0^{*}}{\partial x} (t, x) = \lim_{t_k \to \infty} \frac{\partial \Pi_0}{\partial x} (t_k + t, x)$

В частности в дальнейшем будем предполагать, что функция $\Pi_0 (t, x)$ удовлетворяет условию (2.12) $\frac{\partial \Pi_0 (t, x)}{\partial t} \le 0$ Тогда будет существовать единственная функция $\Pi_0^{*} (x)$

При принятых предположениях уравнения (2.11) предкомпактны и для них можно определить семейство предельных уравнений вида 

\begin{equation}
\frac{d x(t)}{d t} = y(t), \frac{d y(t)}{d t} = A_1^{-1} (x(t)) (C_1 (x(t), y(t)) y(t) - \frac{\partial \Pi_0^{*} (x(t))}{\partial x})- \int_{t- h}^{t} P(t, \tau) (x(t) - x(\tau)) d \tau
\end{equation}

Действительно, в предельном переходе, согласно теореме 1.2 при $t_n \to + \infty$ находим 

$\lim_{t_k \to + \infty} A_1^{-1} x(t_k + t) (C_1 (x(t_k + t), y(t_k + t)) - \frac{\partial \Pi_0 (t_k + t, x(t_k + t))}{\partial x}) + \int_{t_k+t-h}^{t_k+t} P(t_k + t, \tau) (x(t_k + t) - x(\tau) d \tau)
= A_1^{-1} (x^{*} (t)) (C_1 (x^{*}(t), y^{*}(t)) y^{*}(t) - \lim_{t_k \to + \infty} \frac{\partial \Pi_0 (t_k +t, x (t_k + t))}{\partial x} + \lim_{t_k \to + \infty} \int_{t - h}^{t} P(t_n + t, t_n + s) ( x(t_n + t) - x(t_n + s)) ds) = A_1^{-1} ( x^{*} (t)) (C_1 (x^{*} (t), y^{*}(t)) y^{*} (t) - \frac{\partial \Pi_0^{*} (x)}{\partial x} + \int_{t - h}^{t}) P (t, s) (x^{*} (t) - x^{*} (s)) ds,$ так как $P(t_k + t, t_k + s) = P(t, s)$
\end{theorem}

Докажем следующее утверждение. 

\begin{theorem}\label{t-2.11}
Пусть уравнение (2.10) выбрано таким образом, что выполнены условия (2.9) относительно $P(t, \tau)$, а также 

1) зависимость $\Pi_0 (t, x)$ является функцией определенно-положительной $\Pi (t, x) \ge a_1 (\| x \|)$

 2) для некоторого $\mu > 0$ и любого малого $\varepsilon, 0 < \varepsilon < \mu, $ найдется число $\delta = \delta (\varepsilon) > 0$ при котором выполнено неравенство $\| \frac{\partial \Pi_0}{\partial x} \| \ge \delta (\varepsilon) \forall x \in \lbrace \varepsilon \le \| x \| \le \mu \rbrace$

Тогда управляющее воздействие (2.10) решает задачу о стабилизации положения (2.5).

Доказательство

Возьмем функционал Ляпуова в виде 

\begin{equation}
V(t, x_t, y_t) = \frac12 y^{'} (t) A_1(x(t)) y(t) + \Pi_0(t, x(t)) + \frac12 \int_{t-h}^{t} (x(t) - x(\tau))^{'} P(t, \tau) (x(t) - x (\tau)) d \tau
\end{equation}

В силу условий (2.2), (2.9), (2.12) наложенных на $A(q), P(t, \tau), \Pi_0(t, x)$ и условия теоремы для этого функционала имеем оценку

$V(t, x_t, y_t) \ge \alpha_0 \| y(t) \|^2 + a_1 (\| x(t) \|)$
$V(t, x_t, y_t) \le \alpha_1 \| y(t) \|^2 + \Pi_0 (0, x(t)) + \| P \| h \max (\| x(t) - x(s) \|^2, t-h \le s \le t)$

$\| P \|$ - норма матрицы в $R^{n \times n}$

Для производной функционала $V$ в силу (2.1) находим $\dot V = y^{'} (t) A_1 (\alpha(t)) \frac{dy}{dt} + \frac12 y^{'} (t) \frac{d A(x(t))}{dt} y(t) + \frac{\partial \Pi^0 (t, x(t))}{\partial t} + (\dot x (t))^{'} \frac{\partial \Pi^0 (t, x(t))}{\partial t} = y^{'} (t) A_1 (x(t)) ( A_1^{-1} (x(t)) (C_1 (x(t), y(t)) y(t) + Q_1 (t, x(t), y(t)) - \frac{\partial \Pi_0 (t, x)}{\partial x}) - \int_{t - h}^{t} P(t, \tau) (x(t) - x(\tau)) d \tau) + \frac12 (x(t) - x(t))^{'} P(t, t) (x(t) - x(t)) - \frac12 (x(t) - x(t - h))^{'} P(t, t - h) (x(t) - x(t - h)) + \int_{t - h}^{t} \dot x^{'} (t) P(t, \tau) (x(t) - x(\tau)) d \tau + \frac12 \int_{t-h}^{t} (x(t) - x(\tau))^{'} \frac{\partial \Pi_0 (t, x(t))}{\partial t} (x(t) - x(\tau)) d \tau + \frac12 y^{'}(t) \frac{d A(x(t))}{dt} y(t) + \frac{\partial \Pi_0 (t, x(t))}{\partial t} + y^{'}(t) \frac{\partial \Pi_0 (t, x(t))}{\partial x} = y^{'} (t) Q_1 (t, x(t), y(t)) + \frac{\partial \Pi_0 (t, x(t))}{\partial t} - \frac12 (x(t) - x(t - h))^{'} P (t, t - h) (x(t) - x(t - h) + \frac12 \int_{t - h}^{t} (x(t) - x(\tau))^{'} \frac{\partial P(t, \tau)}{t} (x(t) - x(\tau)) d \tau \le - \frac{\beta_0}{2} \int_{t-h}^{t} \| x(t) - x(\tau) \|^2 d \tau \le 0$

Таким образом, для производной $\dot V(t, x_t)$ имеем оценку

\begin{equation}
\dot V(t, x_t) \le - W(t, x_t) \le 0, W(t, x_t) = \frac{\beta_0}{2} \int_{t - h}^{t} \| x(t) - x(\tau) \| ^ 2 d \tau, \lbrace W^{*} (t, x_t) = 0 \rbrace
\end{equation}

Множество $ \lbrace V \equiv {W(t, x_t) = 0 \rbrace$ есть множество функций, для которых $x(\tau) \equiv x(t)$ для всех $\tau \in [t - h, t]$, при всех $t \in R$ и значит 

\begin{equation}
x(t) \equiv x(0) = x_0 = const
\end{equation}

Подставляя $x^{*}(t) = x_0$ в любое предельное уравнение (2.13) получим, что указанное решение должно удовлетворять соотношению 

\begin{equation}
\frac{\partial \Pi^{*} (x^{*} (t))}{\partial x} \equiv 0
\end{equation}

Но это в силу условиия 2) теоремы возможно, если только $x^{*} (t) \equiv 0,$ а значит и $y^{*}(t) \equiv 0.$

Согласно теореме 1.4 имеем искомое доказательство.

Используя оценку производной (1.2.15) функционала (1.2.14) на основании теорем 1.2 и 1.3 получим следующий результат

\begin{theorem}\label{t-1.8}
Допустим, что для уравнения (2.10) при условии (1.9) функция $\Pi_0 (t, x) \ge 0$. Тогда каждое ограниченное для всех $t \ge t_0$ движение системы (2.11) $(\dot x(t), x(t))$ неограниченно приближается к множеству $ \lbrace \dot x = 0, \frac{\partial \Pi_0^{*} (x)}{\partial x} \rbrace = 0$ при $t \to + \infty$
\end{theorem}

Доказательство

Согласно оценке (1.2.15) максимально инвариантное относительно можества $\lbrace W^{*}(t, x_t) \equiv 0 \rbrace$ подмножество состоит из решений (1.2.16), для которых $x^{*} (t) \equiv x_0, \dot x^{*}(t) \equiv 0$ и выполняется равенство (1.2.17). В соответствии с теоремой 1.3 имеем требуемое доказательство.

Теоремы 1.5 и 1.6 позволяют вывести применимость уравнения (2.10) в задаче о стабилизации программной позиции (2.15) по отношению к скоростям и части координат. 

\begin{theorem}\label{t-1.9}
Теорема 1.9. Пусть уравнение (2.10) выбрано таким образом, что выполнены условия (2.9) относительно матрицы $P(t, \tau),$ а также 

1) движения системы (2.11) из некоторой окрестности положения $\dot x = x= 0$ под действием управления (2.10) ограничены по переменным $x_{m+1}, x_{m+2}, ... , x_n$ 

2) функция $\Pi_0 (t, x)$ определенно-положительна по $x_1, x_2, ... x_m, \Pi_0(t, x) \ge a_1 (\| x \|_m), \| x \|^2_m = x_1^2 + x_2^2 + ... + x_m^2$ 

3) вне множества ${x_1 = x_2 = ... = x_m = 0}$ нет поожений равновесия (2.11), так что для некоторого $\mu > 0$ и любого малого числа  $\varepsilon, 0 < \varepsilon < \mu$ найдется $\delta = \delta(\varepsilon) > 0$ такое, что $\| \frac{\partial \Pi_0 (t, x)}{dx} \| \ge \delta(\varepsilon) \forall x \in \lbrace \varepsilon \le \sum_{k = 1}^{m} x_k^2 \le \mu \rbrace$

Тогда программная позиция (2.5) или положение $\dot x = x = 0$ системы (2.10) асимптотически устойчиво по $\dot x, x_1, x_2,.., x_m.$
\end{theorem}

Доказательство 

Вновь возьмем функционал Ляпунова в виде (1.2.14). В силу уравнения 2) имеем оценку

\begin{equation}
V(t, x_t, y_t) \ge \alpha_0 \| y \| ^2 + a_1 (\| x(t) \|_m)
\end{equation}

Для производной этого функционала по прежнему имеем оценку (1.2.15) и, таким образом, имеет место устойчивость по $y, x_1, ... x_m$

В силу условия 3) теоремы множество 

$ \lbrace W^{*} (t, x_t) = 0 \rbrace \cap \lbrace x : \sum_{k = 1}^{m} x_k^2 > 0 \rbrace = \lbrace x(\tau) = x(t), t - h < \tau < t \rbrace \cap \lbrace \sum_{k = 1}^{m} x_k^2 > 0 \rbrace = \lbrace x(t) = c = const \rbrace \cap \lbrace \sum_{k = 1}^{m} x_k^2 > 0 \rbrace = \lbrace x: \frac{\partial \Pi^{*} (x)}{\partial x} = 0 \rbrace \cap \lbrace \sum_{k = 1}^{m} x_k^2 > 0 \rbrace$

не содержит решений предельной системы. В соответствие с теоремой 1.8 для каждого ограниченного движения $y(t), x(t)$ имеем.

$\lim_{t \to + \infty} y(t) = 0; \lim_{t \to + \infty} x_k (t) = 0, k = 1..m$

Теорема доказана

\begin{theorem}\label{t-1.11}
Пусть уравнение (2.10) выбрано таким образом, что выполнены условия (2.9) относительно матрицы $P(t, \tau),$ а также: 

1) движения системы (2.11) из некоторой окрестности положения $\dot x = x = 0$ равномерно ограничены по переменным $x_{m+1}, x_{m+2},... x_n$

2) функция $\Pi_0 (t, x)$ удовлетворяет неравенствам $a_1 (\| x \| _m) \le \Pi_0 (t, x) \le \Pi^{*} (x)$

3) множество $\lbrace \Pi^{*} (x) > 0 \rbrace$ не содержит положений равновесия системы (2.13), $\| \frac{\partial \Pi_0^{*} (\dot x)}{\partial x} \| \ge \delta(\varepsilon) > 0 \forall x \in \lbrace \Pi^{*} (x) \ge \varepsilon \rbrace$

Тогда программная позиция (2.5) или положение равновесия $\dot x = x = 0$ системы (1.2.10) равномерно асимптотически устойчиво по $\dot x, x_1, x_2, ... x_m.$
\end{theorem}

Доказательство

Вновь используем функционал Ляпунова (1.2.14), для которого, согласно условию теоремы, имеем оценку $a_1 (\| x(t) \|) + \alpha_0 \| y(t) \|^2 \le V(t, x_t, y_t) \le V_1 (x_t, y_t) = \alpha_1 \|y(t)\|^2 | \Pi^{*} (x(t)) + \frac12 \| P \| h \| x(t) - x(\tau) \|^2$

Согласно соотношению (1.2.15) для производной этого функционала находим, что множество $\lbrace W^{*} (t, x_t) = 0 \rbrace \cap \lbrace V_1(x_t, y_t) > 0 \rbrace = \lbrace x(\tau) = x(t), t-h \le t \le t \rbrace \cap \lbrace \alpha_1 \| y(t) \|^2 + \Pi^{*} (x(t)) + \frac12 \| P\| h \| x(t) - x(\tau) \|^2 > 0 \rbrace \equiv \lbrace \Pi^{*} (x(t)) > 0 \rbrace$

не может в соответствии с условием 3) теоремы содеражать решение предельной системы. Согласно теореме 1.6 имеем требуемое доказательство.

Замечание. Условиея 1) теорем 1.8 и 1.9 заведомо выполняются, если переменные $q_{m+1}, ... q_n$ являются периодическими, например по $mod 2 \pi: q_k + 2 \pi = q, k = m + 1, n$