\section{О стабилизации стационарных программных движений управляемой механической системы.} \label{p13}

Рассмотрим механическую систему, у которой кинетическая энергия не зависит от последних $(n-m)$ координат $q_{m+1},... q_n (0 < m < n, n > 1),$ и, таким образом, первые $m$ координат $q_1, q_2, ... , q_m$ являются периодическими, остальные - циклическими [].

Переобозначим координаты, положим

$z = (z_1, z_2, ... z_m)^{'} = (q_1, q_2, ..., q_m)^{'}$

$s^{'} = (s_1, s_2, ... , s_{n-m}) = (q_{m + 1}, ..., q_n)^{'}$

Соответственно, кинетическая энергия системы может быть записана в виде $2T = \dot z^{'} A_{11} (z) \dot z + \dot z^{'} A_{12} (z) \dot s + \dot s^{'} A_{21} (z) \dot z + \dot s^{'} A_{22} (z) \dot s, \quad A_{21} = A_{12}$

где составляющие матрицу $A$ подматрицы $A_{11}, A_{12}, A_{21}, A_{22}$ удовлетворяют условиям 
$$
\begin{array}{l}
\alpha_0 \| z \|^2 \le z^{'} A_{11} z \le \alpha_1  \| z \|^2\\
\alpha_0 \| s \|^2 \le s^{'} A_{22} s \le \alpha_1  \| s \|^2\\
\alpha_0 \| s \|^2 \le s^{'} B s \le \alpha_1  \| s \|^2\\
B = A_{12} A_{22} A_{21}, \alpha_0, \alpha_1 > 0 - const\\
\end{array}
$$

Будем полагать, что обобщенные силы и управление по циклическим координатам отсутствует

\begin{equation} \label{1.31'}
Q^{'} = (Q_{z}^{'}, \quad Q_{s}^{'}), \quad U = (U_{z}^{'}, U_{s}^{'}), Q_s = 0, \quad U_s = 0, \quad \frac{\partial \Pi}{\partial s} = 0
\end{equation}

Введем циклические импульсы

\begin{equation} \label{1.32'}
v = \frac{\partial T}{\partial \dot s} = A_{21} (z) \dot z + A_{22} (z) \dot s
\end{equation}

Из последних $n - m$ уравнений (\ref{1.14'}) и предположений (\ref{1.31'}) получаем 

\begin{equation} \label{1.33'}
\frac{d v}{d t} = 0
\end{equation}

т.е. циклические импульсы для движения системы (\ref{1.14'}) являются постоянными.

Следуя [], введем функцию Рауса. Для этого из (\ref{1.32'}) находим

\begin{equation} \label{1.34'}
\dot s = A_{22}^{-1} (z) (v - A_{21} (z) \dot z)
\end{equation}

Тогда функция Рауса [] 

\begin{equation} \label{1.35'}
\begin{array}{c}
	2R = 2T - 2 \dot s^{'} v =\\
	= \dot z^{'} A_{11} \dot z + \dot z^{'} A_{12} A_{22}^{-1} (v - A_{21} \dot z) + (v - A_{21} \dot z)^{'} A_{21} \dot z +\\
	+ (v - A_21 \dot z)^{'} A_{22}^{-1} (v - A_{21} \dot z) - 2 (v - A_{21} \dot z)^{'} A_{22}^{-1} P =\\
	= 2 R_2 + 2 R_1 + 2 R_0, 2 R_2 = \dot z^{'} B (z) \dot z, R_1 = v A_{22}^{-1} A_{21} \dot z, 2 R_0 = - v^{'} A_{22}^{-1} v
\end{array}
\end{equation}

При этом матрица $B(z)$ является опредленно-положительной.

Уравнения движения системы (\ref{1.14'}) могут быть записаны в виде равенств (\ref{1.33'}) и (\ref{1.34'}) и уравнений

\begin{equation} \label{1.36'}
\frac{d}{dt} (\frac{\partial R}{\partial \dot z}) - \frac{\partial R}{\partial z} = - \frac{\partial \Pi}{\partial z} + Q_z + U_z
\end{equation}

После подстановки выражения для функции $R$ получаем следующие уравнения движения 

\begin{equation} \label{1.37'}
\begin{array}{c}
\displaystyle \frac{d}{dt} (\frac{\partial R_2}{\partial \dot z}) - \frac{\partial R_2}{\partial z} = \frac{\partial R_0}{\partial z} - \frac{\partial \Pi}{\partial z} - G \dot z + U_z + Q_z,\\
{d v}{d t} = 0,\\
\displaystyle \dot s = A_{22}^{-1} (z) (v - A_{21} (z) \dot z)
\end{array}
\end{equation}

где $G = G(z, v) = (\frac{\partial g}{\partial z}) - (\frac{\partial g}{\partial z})^{'}, g(z, v) = A_{12} A_{22}^{-1} v$ есть кососимметрическая матрица, $G^{'} = -G$.

Допустим, что при некотором значении $z = z_0$ и $v = v_0$ управление $U_z$ подобрано так, что 

\begin{equation} \label{1.38'}
U_z^0 (t) = - \frac{\partial R_0}{\partial z} (v_0, z_0) + \frac{\partial \Pi (t, z_0)}{\partial z}
\end{equation}

Тогда система (\ref{1.35'}) будет иметь программное движения 

\begin{equation} \label{1.39'}
\dot z = 0, \quad z = z_0, \quad v = v_0, \quad \dot s = \dot s_0 = A_{22}^{-1} (z_0) v_0
\end{equation}

Рассмотрим задачу о стабилизации этого движения управлением вида 

\begin{equation} \label{1.40'}
U_z = U_z^0 (t) - \frac{\partial \Pi_U (t, v, z)}{\partial z} - \int_{t - h}^{t} P_z (t, \tau) (z(t) - z(\tau)) d \tau
\end{equation}

где $\Pi_U = \Pi_U (t, v, z)$ есть некоторые непрерывно-дифференцируемая (по $z$ дважды) ограниченная вместе со своими производными $\frac{\partial \Pi}{\partial t} (t, v, z_0) 0 , P_z(t, z)$ есть матрица размерности $m \times m$ выражения вида $P_z = \| p_{jk} (t, \tau) \|, p_{jk} (t, \tau) = p_{jk}^0 e^{s_{jk}^0 (\tau - t)}, p_{jk}^0, s_{jk}^0 - const$

удовлетворяющая условиям

\begin{equation} \label{1.41'}
\begin{array}{c}
\dot z P(t, \tau) z \ge 0, z^{'} P_t (t, \tau) z \le - \beta_0 \| z \|^2, \beta_0 > 0,\\
P_t (t, \tau) = \frac{\partial P_z (t, \tau)}{\partial t} = - \| p_{jk}^0 s_{jk}^0 e^{s_{jk}^0 (\tau - t)} \|,\\
\| z \| ^ 2 = z_1^2 + ... + z_m^2
\end{array}
\end{equation}

Уравнения (\ref{1.36'}) с управлением (\ref{1.40'}), (\ref{1.41'}) могут быть записаны в виде

\begin{equation} \label{1.42'}
\begin{array}{c}
\displaystyle \frac{d z (t)}{dt} = r(t),\\
\displaystyle \frac{d r(t)}{dt} = B^{-1} (z(t)) (E(z(t), r(t)) r(t) - G(z(t), v(t)) r(t) + Q_z (t, z(t), r(t)) -\\
\displaystyle - \frac{\partial S(t, v(t), z(t))}{\partial z}) - \int_{t - h}^{t} P_z (t, \tau) (z(t) - z(\tau)) d \tau,
\end{array}
\end{equation}
$$\frac{d v(t)}{dt} = 0 \quad \frac{d s(t)}{dt} = A_{22}^{-1} (z(t) (v(t) - A_{21}) (z(t)) r(t))$$

где $E = (e_{kj}) - $ матрица, определяемая равенством 
$$e_{kj} = \frac12 \sum_{ i = 1}^{m} (\frac{\partial b_{ik}}{\partial z_j} - \frac{\partial b_{kj}}{\partial z_i} - \frac{\partial b_{ij}}{\partial z_k}) r_k$$
функция $S(t, v, z) = \Pi (t, z) - R_0 (v, z) - \frac{\partial \Pi (t, z_0)}{\partial z} z + \frac{\partial R_0 (v, z_0)}{\partial z} z + \Pi_U (t, v, z)$

В силу подбора $U_z$ имеем $\frac{\partial S(t, v_0, z_0)}{\partial z} = 0$, так что система (\ref{1.42'}) имеет решение 
$$z(t) = z_0, \quad r(t) = 0, \quad v = v_0, \quad \dot s_0 = A_{22}^{-1} (z_0) v_0,$$ 
отвечающее программному движению (\ref{1.39'}).

Полагаем, что функции $Q_z (t, z, r)$, $S(t, v, z)$, $\frac{\partial S}{\partial t} (t, v, z)$, $\frac{\partial S(t, v, z)}{\partial z}$ являются ограниченными, равномерно непрерывными на множествах вида $R \times K, K \in R^n$ - компакт, при этом

\begin{equation} \label{1.43'}
\frac{\partial S}{\partial t} (t, v, z) \le 0, \quad S(t, v_0, z_0) = 0
\end{equation}

Уравнения (\ref{1.42'}) имеют аналогичный им вид с функциями $Q_z^{*} (t, z, r)$ и $S = S^{*} (v, z),$ предельными к $Q_z$ и $S$
$$S^{*} (v, z) = \lim_{t \to + \infty} S(t, v, z).$$
Введем функционал
$$
\begin{array}{c}
\displaystyle V = \frac12 r^{'}(t) B (z(t)) r(t) + S(t, v_0, z) + \frac12 \| v(t) - v_0 \|^2 +\\
\displaystyle + \frac12 \int\limits_{t - h}^{t} (z(t) - z(\tau))^{'} P_z (t, \tau) (z(t) - z(\tau)) d \tau
\end{array}
$$

Его производная в силу уравнений (\ref{1.42'}) в соответствии с соотношениями (\ref{1.41'}) и (\ref{1.43'}) будет иметь следующую оценку

\begin{equation} \label{1.44'}
\begin{array}{c}
\displaystyle \dot V = - z^{'} (t) G(z(t), v(t)) r(t) + z^{'} (t) Q_z (t, z(t), r(t)) + \frac{\partial S(t, v_0, t)}{\partial t} +\\
\displaystyle + \frac12 \displaystyle\int\limits_{t - h}^{t} (z(t) - z(\tau))^{'} P_t(t, \tau) (z(t) - z(\tau)) d \tau \le - W(z_t) =\\
\displaystyle = -\frac{\beta_0}{2} \displaystyle\int\limits_{t - h}^{t} \| z(t) - z(\tau) \|^2 d \tau
\end{array}
\end{equation}

Соответственно имеем следующую теорему.

\begin{theorem}\label{t-1.10}
Предположим, что:

1) функция $s = s(t, v_0, z)$ является определенно-положительной по $z$;

2) функция $s^{*} (v_0, z)$ имеет изолированный минимум в точке 
$$\| \frac{\partial S^{*} (v_0, z)}{\partial z} \| \ne 0 \quad \forall z \in \lbrace 0 < \| z - z_0 \| < \delta \rbrace$$

Тогда под действием управления (\ref{1.40'})) программное стационарное движение (\ref{1.39'}) устойчиво, асимптотически устойчиво относительно движений, отвечающих значению $v = v_0,$ а каждое иное возмущенное движение неограниченно приближается при $t \to + \infty$ к движению 
$$\dot z = 0, \quad \dot z = z_1 = const, \quad v = v_1 = const, \quad \dot s = \dot s_1 = A_{22}^{-1} (z_1) v_1$$
\end{theorem}

Доказательство:
Из условия 1) теоремы и условий относительно функций, входящих в функционал $V,$ следует, что он является определенно положительным, допускающим бесконечно малый высший предел по переменным $r, z, v - v_0.$

В силу (\ref{1.44'}) имеем устойчивость движения (\ref{1.39'}).

Множество $\lbrace W(z_t) = 0 \rbrace$ может содержать только те предельные решения, для которых $z(t) = z_1 = const.$ Но такими решениями могут быть решения для которых одновременно 
$$r(t) = 0, \quad v(t) = v_1 = const, \quad \frac{\partial S^{*} (v_1, z_1)}{\partial z} = 0.$$
Тем самым, в силу теорем 1.1 и 1.2 имеем искомое доказательство. 