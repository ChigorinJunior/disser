\section{Стабилизация программных движений двузвенного манипулятора без измерения скоростей.} \label{p22}

Сначала рассмотрим задачу о стабилизации программного положения горизонтального манипулятора. В этом случае без измерения скоростей 

\begin{equation}
q^0_1 (t) = q^0_1 = const, q^2_0 (t) = q^2_0 = const
\end{equation}

В соответствии с представленным в параграфе 2.2 общим решением для голономной механической системы эта задача решается управлением вида 

\begin{equation}
U^{(1)}_1 (x_1) = - k_1 \sin \frac{x_1(t)}{2} - \int_{t-h}^t p_1^0 e^{S_1^0 (\tau - t)} (x_1 (t) - x_1 (\tau)) U^{(2)}_1 (x_1) = - k_2 \sin \frac{x_2(t)}{2} - \int_{t-h}^t p_2^0 e^{S_2^0 (\tau - t)} (x_2 (t) - x_2 (\tau))
\end{equation}

$k_1, k_2, p_1^0, p_2^0 - const$

При этом, согласно теореме каждое возмущенное движение будет неограниченно приближаться при $t \to \infty$ к положению равновесия, определяемое равенствами 

$$\sin \frac{x_1(t)}{2} = 0 \sin \frac{x_2(t)}{2} = 0$$

или $x_1(t) = \pi k, x_2(t) = 2 \pi k, k \in Z$

Соответственно в переменных $q_1$ и $q_2$ для каждого возмущенного движения имеем при $t \to +\infty$ 

\begin{equation}
q_1 (t) \to q_1^0 + 2 \pi k, q_2(t) \to q_2^0 + 2 \pi k
\end{equation}


Таким образом, так как положение манипулятора в переменных $q_1$ и $q_2$ определяются с точности до $2 \pi,$ управлением (2.2.2) достигается глобальная стабилизация положения (2.1.1)

Выражение для кинетической энергии не содержит координату $q_1$ в явном виде, т.е. она является циклической. Вводим циклический импульс $\frac{\partial T}{\partial \dot q_1} = a_{11} \dot q_1 + a_{12} \dot q_2$

Составим функцию Рауса. Для этого находим $\dot q_1 = (p - a_12 \dot q_2) / a_{11}$

$R = \frac12 (p - a_{12} \dot q_2)^2 / a_{11} + a_{22} (p - a_{12} \dot q_2) \dot q_2 / a_{11} + \frac12 a_{22} \dot q_2^2 - p (p - a_{12} \dot q_2) / a_{11} = \frac12 (a_22 - a_{12}^2 / a_{11}) \dot q_2^2 - \frac12 p^2 / a_{11} + a_{12} p \dot q_2 / a_{11}$

Отсюда имеем следующие уравнения движения в переменных $q_2, \dot q_2, p, p = p_0 = const$

$\frac{d}{dt} ((a_{22} - a_{12}^2 / a_{11}) \dot q_2) - \frac12 \frac{\partial a_{11}}{\partial q_2} p_0^2 / a_{11}^2 = M_1$


Эти уравнения допускают движения вида 

\begin{equation}
p = p_0 = const, \dot q_2 = 0, q_2 = q_2^0 = const
\end{equation}

если $U_1^0 = m_2 l l_2 \sin q_2^0 p_0^2 / (m_2 l^2 + I_1 + I_2 + 2 m_2 l l_2 \cos q_2^0)^2$

В соответствии с представленным в параграфе 2.3 общим решением в задаче о стабилизации установленного движения голономной механической системы с циклической координатой получаем, что задача о стабилизации движения (2.2.4) по $q_2$ и $\dot q_2$ решается уравнением 

$U_2 = U_2^0 - k_2 \sin x_2 (t) - \int_{t-h}^{t} p_2^0 e^{S_2^0 (\tau - t)} (x_2 (t) - x_2 (\tau)) d \tau$

$k_2 = \frac{m_2 l l_2 \cos q_2^0 p_0^2}{(m l^2 + I_1 + I_2 + 2 m_2 l l_2 \cos q_2^0)^2}; p_2^0,  p_2^0 - const > 0$

может быть достигнута также стабилизацией движения (2.2.4) по $p, \dot q_2$ и $q_2.$

Рассмотрим задачу о стабилизации заданного программного положения двузвенного манипулятора. Из уравнений движения (2.2.3) находим, что положение (2.2.1) будет иметь место управление $U_1^0 = (m_1 l_1 + m_2 l_2) g \cos q_1^0, U_2^0 = m_2 l_2 g \cos (q_1^0 + q_2^0)$

В соответствии в параграфе 1.2 общим решением получаем, что задачи о стабилизации программного положения (2.2.1) без измерения скоростей решается управлением 

$U_1 = U_1^0 - k_1 (q_1(t) - q_1^0) - \int_{t-h}^{t} p_1^0 e^{S_1^0 (\tau - t)} (q_1(t) - q_1(\tau)) d \tau$

$U_2 = U_2^0 - k_2 (q_2(t) - q_2^0) - \int_{t-h}^{t} p_2^0 e^{S_2^0 (\tau - t)} (q_2(t) - q_2(\tau)) d \tau$

где коэффициенты $k_1$ и $k_2$ удовлетворяет условиям 

$\alpha_1 k_1 - (m_1 l_1 + m_2 l) g \sin q_1^0 - m_2 l_2 g \sin (q_1^0 + q_2^0) > 0$

$\alpha_1 (k_2 - m_2 l_2 g \sin (q_1^0 + q_2^0)) - (m_2 l_2 g \sin (q_1^0 + q_2^0))^2 > 0$