{\centerline{ {\sc ЗАКЛЮЧЕНИЕ }} \label{postfix}
\paragraph{}
В представленной диссертационной работе разработаны новые методы стабилизации нелинейных управляемых систем, в частности многозвенных манипуляторов. Представлены новые методы исследования устойчивости.

Ниже представлены основные результаты работы.

\begin{enumerate}
	\item{Развитие метода векторных функций Ляпунова для исследования устойчивости нелинейных систем.}
	\item{Полученные результаты являются развитием соответствующих теорем из работ. Эффективность новой методики в исследовании и стабилизации представлена на примере решения задач стабилизации}
	\item{Полученные результаты являются развитием для дискретных систем соответствующих теорем 
		из работ \cite{andr971, andr124, andr055} для систем, описываемых дифференциальными уравнениями, обобщением результатов 
		работ из \cite{bogd1404, bogdanovmono, bromberg67, kunzevi77, martynuk00, lakshm02, lasalle76, lasalle77, yang01}.  
		Эффективность новой методики в исследовании устойчивости и стабилизации представлена на примере решения задачи 
		об устойчивости системы, моделируемой уравнениями типа Вольтерра. Получено решение задачи о стабилизации программных 
		движений управляемых механических систем со ступенчатым импульсным управлением.  Построены соответствующие модели управления 
		системой с одной и многими степенями свободы, с одной позиционной и остальными циклическими координатами. Эти результаты 
		представляют собой развитие для дискретного управления соответствующих результатов о стабилизации программных движений 
		механических систем посредством непрерывных и релейных управлений из работ  \cite{andr126, andr078, andr111, andr044, andr14vspu,
			vukobr82, matuhun01, peregudova09, peregudova092, peregudova07, peregudova13, pjatnic873, pjatnic882, pjatnic937, finogenko07, jurkov01}. }
	\item{Представлены модели двух- и трехзвенных манипуляторов, модель перевернутого маятника}
	\item{Разработана компьютерная модель движения трехзвенного манипулятора на неподвижном основании. Программный комплекс разработан на языке Java с использованием фреймворка JavaFX. Приложение является кроссплатформенным, позволяет задавать законы управления в аналитическом виде.}
\end{enumerate}

Основные результаты работы опубликованы в работах, в том числе, в статьях из списка ВАК. Получен патент РФ на программу для ЭВМ на программу моделирования движения трехзвенного манипулятора.


