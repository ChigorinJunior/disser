{\centerline{ {\sc ЗАКЛЮЧЕНИЕ }} \label{postfix}
\paragraph{}
В диссертационной работе разработаны новые методы исследования устойчивости и стабилизации нелинейных управляемых систем, 
моделируемых дискретными уравнениями; 
обоснована методика построения моделей дискретного  управления движениями  управляемых механических систем.
Основные результаты работы состоят в следующем.

\begin{enumerate}
	\item{Проведено развитие метода векторных функций Ляпунова а исследовании устойчивости и стабилизации нелинейных систем, моделируемых дискретными уравнениями.}
	\item{Полученные результаты являются развитием для дискретных систем соответствующих теорем 
	из работ \cite{andr971, andr124, andr055} для систем, описываемых дифференциальными уравнениями, обобщением результатов 
	работ из \cite{bogd1404, bogdanovmono, bromberg67, kunzevi77, martynuk00, lakshm02, lasalle76, lasalle77, yang01}.  
	Эффективность новой методики в исследовании устойчивости и стабилизации представлена на примере решения задачи 
	об устойчивости системы, моделируемой уравнениями типа Вольтерра. Получено решение задачи о стабилизации программных 
	движений управляемых механических систем со ступенчатым импульсным управлением.  Построены соответствующие модели управления 
	системой с одной и многими степенями свободы, с одной позиционной и остальными циклическими координатами. Эти результаты 
	представляют собой развитие для дискретного управления соответствующих результатов о стабилизации программных движений 
	механических систем посредством непрерывных и релейных управлений из работ  \cite{andr126, andr078, andr111, andr044, andr14vspu,
	vukobr82, matuhun01, peregudova09, peregudova092, peregudova07, peregudova13, pjatnic873, pjatnic882, pjatnic937, finogenko07, jurkov01}. }
	\item{Представлена модель управляемого двузвенного манипулятора на подвижном основании со ступенчатым импульсным управлением.}
	\item{Разработана компьютерная модель управляемого движения колесного мобильного~робота~с~тремя~омни-колесами. Разработанный программный комплекс на языке высокого уровня Java с самостоятельным кроссплатформенным приложением позволяет составить  анализ процесса управления при различных способах его задания – в виде функций, в виде поточечного закона и их модификаций.}
\end{enumerate}	
Основные результаты работы опубликованы 
в работах \cite{kudash071,  kudash0946,  kudash1013, kudash1034, kudash119,  kudash12, kudash0967, kudash122, kudash14ek, kudash14su} 
в том числе, в статьях \cite{andr136, kudash136, kudash145, artyomova135, kudash0916, kudash092, kudash131, kudash151, kudash152 } 
опубликованных в журналах из списка ВАК. На программу моделирования управляемого движения мобильного робота получен патент РФ 
на программу для ЭВМ \cite{kudashpat}.


