{\centerline{ {\sc ВВЕДЕНИЕ }} \label{prefix}
	Развитие производства в XX веке повлекло за собой совершенствование средств автоматизации. Использование всевозможных управляемых механизмов повлекло за собой необходимость в развитии математического описания их функционирования для обеспечения оптимальности выполняемых операций. Важная роль в вопросе управляемости механических устройств отведена манипуляторам, как средству выполнения роботом необходимой задачи. 
	Происходит постепенное движение от более простых моделей к более сложным, позволяющим учитывать нелинейную движений манипуляторов, решать задачи с запаздыванием, возникающим в цепи обратной связи.
	Важная роль уделяется также снижению вычислительной сложности расчётов заданного движения. Данное условие необходимо для возможности просчёта движений онлайн, что обеспечивает большую гибкость возможностей использования манипуляторов.
	Так как модели механических систем часто представляют собой системы нелинейных дифференциальных уравнений, одним из естественных вариантов решения задачи анализа поведения таких систем становится метод декомпозиции, позволяющей проводить разбиение системы на подсистемы меньшей размерности для их дальнейшего исследования. Методы декомпозиции развиваются научными школами таких учёных как Ф.Л. Черноусько и Е.С. Пятницкий. 
	Проблема управления движением механических систем, в том числе, манипуляционных роботов, без учета измерения скоростей стала активно изучаться с начала 90-х годов прошлого века. В ранних исследованиях \cite{nicosia, berghuis0, kelly, berghuis1, berghuis2} были получены результаты, решающие задачи стабилизации программной позиции и локального отслеживания траектории. Эти результаты, как правило, были основаны на применении двухшаговой процедуры: 1) построение наблюдателя (фильтра) скоростей; 2) синтез управления с применением метода линеаризации обратной связью и функции Ляпунова квадратичного вида. Такие законы управления являются весьма сложными по структуре, так как содержат вычисляемые в режиме он-лайн моменты всех сил, действующих на систему, слагаемое, представляющее собой произведение матрицы инерции системы на программное ускорение. Точная реализация данных законов возможна лишь на имеющейся полной информации о параметрах системы и действующих силах. В работах \cite{loria0, loria1} решены задачи полуглобального и глобального отслеживания траектории механической системы с одной и с n степенями свободы без учета измерения скорости на основе применения приближенного дифференцирования и построения управления при помощи метода линеаризации обратной связью. Как отмечалось ранее, недостатком данного метода является сложность структуры построенного управления, большие объемы вычисления в режиме он-лайн и необходимость построения точной динамической модели системы. В работах \cite{burkov0, burkov1} для решения задач стабилизации программной позиции и программного движения натуральной механической системы без измерения скоростей были построены наблюдатели, имеющие порядок, равный числу степеней свободы системы, не требующие точной информации о динамической модели системы, что является преимуществом перед нелинейными наблюдателями, предложенными в работах \cite{nicosia, berghuis0}. Результаты, полученные в работах \cite{burkov0, burkov1}, применимы лишь для механических систем без учета диссипативных сил, кроме того, решение задачи о стабилизации программного движения получено в малом, что ссужает область применимости данных результатов. В работе \cite{loria2} дано решение задачи полуглобального отслеживания траектории механических систем, находящихся под действием лишь потенциальных и ограниченных управляющих сил, что ссужает класс рассматриваемых механических систем. В работе \cite{calugi} на основе применения классического метода Ляпунова построено адаптивное управление многозвенным манипулятором на основе наблюдателя и применения метода бэкстеппинга без измерения скоростей и с учетом неизвестных параметров системы. В работе \cite{alonge} было получено адаптивное управление манипуляционным роботом без измерения скоростей с использованием фильтров первого порядка. Недостатком работ \cite{calugi, alonge} является сложная структура построенного управления. В работе \cite{dixon} предложен робастный закон управления, решающий задачу глобального отслеживания траектории робота манипулятора с неточно известными параметрами без измерения скоростей, недостатком работы является сложный алгоритм построения управления, требующий большого объема вычислений в режиме он-лайн. В работе \cite{nunes} даны решения задач управления нелинейных механических систем под действием диссипативных сил без измерения скоростей с гравитационным компенсатором: о глобальной стабилизации программной позиции на основе динамической обратной связи с насыщением; глобального отслеживания траектории. При этом нерешенной остается задача построения закона управления без гравитационного компенсатора, в том числе, для систем с неточно известными параметрами. В работе \cite{burkov2} решена задача о глобальной стабилизации программной позиции механической системы, находящейся под действием лишь потенциальных и управляющих сил. С помощью нелинейной обратной связи построен закон управления без измерения скоростей. При этом вопрос о робастности построенного закона не рассматривался. Нерешенной остается задача о стабилизации нелинейной обратной связью программного движения более широкого класса механических систем, находящихся под действием не только потенциальных, но и диссипативных сил. В работе \cite{yarza} построен закон адаптивного управления, обеспечивающего равномерную глобальную асимптотическую устойчивость заданного движения манипуляционного робота. На основе классического метода Ляпунова и построения нелинейных фильтров задача адаптивного управления решена для механических систем с линейной зависимостью от вектора неизвестных параметров. Отметим, что для реализации построенного в \cite{yarza} закона требуется проведение громоздких вычислений для построения оценки неизвестных параметров, кроме того, открытым остается вопрос оценки скорости сходимости к программному движению. В работе \cite{loria3} решена задача об отслеживании нестационарной траектории механических систем без измерения скоростей и без построения наблюдателей. При этом для нахождения неизвестных скоростей применяется приближенное дифференцирование. Получены условия равномерной глобальной асимптотической устойчивости программного движения системы без диссипации путем построения нелинейного закона управления на основе метода линеаризации обратной 
	связью. Проведенный анализ работ \cite{nicosia} - \cite{loria3} позволяет утверждать, что к настоящему моменту решение задачи о нелокальной стабилизации нестационарных программных движений нелинейных механических систем с неточно известными параметрами без измерения скоростей далеко от завершения.
	
	В 80-х гг. из-за развития вычислительной техники, учёные при разработке методов моделирования начинают ориентироваться на эффективность метода с точки зрения применимости для решения на компьютере.  Метод оценивается с помощью О – символики. Алгоритмы, разработанные с применением метода Лагранжа, имели сложность $O(N^4)$ и должны были быть адаптированы для систем, работающих в реальном времени. Первые исследователи, разработавшие методы порядка $O(N)$ для решения обратной задачи динамики использовали уравнения Ньютона-Эйлера. Так Степаненко и Вукобратович в 1976-м разработали рекурсивный метод Ньютона-Эйлера для описания динамики человеческих конечностей. Орин в 1979 году улучшил этот метод, привязав силы и моменты сил к локальным координатам звеньев для контроля ноги шагающей машины в реальном времени. Лу, Уокер и Пол в 1980-м разработали очень эффективный рекурсивный алгоритм на основе уравнений Ньютона-Эйлера (RNEA), привязав все основные параметры к координатам звеньев. Холлербах, разработавший в том же году алгоритм на основании уравнений Лагранжа, однако, оказалось, что он намного менее эффективен, чем RNEA в терминах количества умножений и сложений. Рекурсивные преобразования и формулы Родрига использовали Вукобратович и Потконяк [И29] (1979 г.), причем их метод позволял решать и прямую задачу динамики, хотя его вычислительная эффективность и не столь высока. Значительный прогресс в сокращении числа операций достигнут в работах Ренода [И24] (1983 г.) и Ли [И15] (1988 г.), также применивших рекурсивные соотношения.[И1]
	
	Самый ранний первый известный алгоритм сложности $O(N)$ для решения прямой задачи динамики был предложен Верещагиным. Этот алгоритм использует рекурсивную формулу для расчета формы Гиббса-Аппеля уравнения движения, и применим к неразветвлённым цепям с вращательными или призматическими соединениями. 
	В 1988 Балафотисом, Пателем и Митрой были представлены дварекурсивных алгоритма на основе уравнений Ньютона-Эйлера, использующие ортогональные тензоры для решения обратной задачи динамики. Они применимы для разомкнутой кинематической цепи (например, для моделирования движения человеческой руки). Один из этих алгоритмов позволяет рассчитать положение манипулятора с шестью степенями свободы, используя приблизительно 489 операций умножения и 420 сложения. Для манипуляторов с более простой геометрической конфигурацией количество операций может быть уменьшено до 277 и 255 соответственно. [И19] Этот алгоритм приблизительно в 1.7 раз быстрее, чем алгоритм RNEA для манипулятора робота с шестью степенями свободы.
	
	В 1954 г. были проведены  работы по симуляции эволюции Нильсом Баричелли на компьютере, установленном в Институте Продвинутых Исследований Принстонского университета. Его работа, опубликованная в том же году, привлекла широкое внимание общественности. [И20] С 1957 года, австралийский генетик Алекс Фразер опубликовал серию работ по симуляции искусственного отбора среди организмов с множественным контролем измеримых характеристик. Симуляции Фразера включали все важнейшие элементы современных генетических алгоритмов. [И21] С ростом исследовательского интереса существенно выросла и вычислительная мощь настольных компьютеров, это позволило использовать новую вычислительную технику на практике. В том числе применять генетические алгоритмы для нахождения оптимального управления робототехническими системами (например, симуляции походки человека). Одна из последних на данное время  работ в этом направлении - работа, представленная в 2013 г. В ней Гейтенбик, Ван де Панн и Ван дер Стаппен продемонстрировали метод моделирования двуногой ходьбы с использованием генетических алгоритмов. [И22] Их метод основывается на моделировании конечностей существа в виде связанных тел, управляемых мышцами, которые могут перемещать конечности определенным образом и основывается на работе Ванга 2012 г. [И30] Проведенные эксперименты показывают, что в обычных условиях модели приходят к стабильному положению ходьбы через 500-3000 поколений. Недостатками, однако, является небольшая производительность (на персональном компьютере время оптимизации 2-12 часов). Также для метода, как в целом для всех генетических алгоритмов характерна сходимость решения к локальному минимуму, что может не обеспечивать достаточной эффективности.
	
	В первой главе диссертации представлены исследования относительно уравнений с запаздыванием.
	Результаты, полученные в первой главе далее применяются для построения управления манипуляторами во второй и третьей главах.
	
	Во второй главе строится и обосновывается управление двухзвенным манипулятором, моделируемым в виде системы связанных твёрдых тел. Рассматривается модель манипулятора, описываемая уравнениями Лагранжа 2-го рода. В первом параграфе ислледуется поведение манипулятора без учёта динамики приводов, рассположенных в шарнирах и приводящих манипулятор в движение. Во втором параграфе система рассматривается с учётом динамики приводов.
	
	Третья глава начинается с построения модели, описывающей поведения трехзвенного манипулятора, имеющего 3 степени свободы, не учитывающей действия электрических приводов. Во втором параграфе рассматривается модель манипулятора с приводами.
	
	Приложение содержит компьютерную модель динамики трехзвенного манипулятора и реализацию данной модели на языке высокого уровня Java. Представленная программа позволяет задавать закон управления в аналитическом виде, в том числе в виде определенных интегралов с переменным верхним пределом.
	
	\newpage
